\chapter*{\centering{\large{ABSTRACT}}}

\begin{spacing}{1}
\textbf{GIAN CHIESA MAGHRIZA}. Design of a Android Based Application to Support Modern Fisheries Technology. Mini Thesis. Computer Science Department, Faculty of Mathematics and Natural Siences, State University of Jakarta. 2023. Under the guidance Muhammad Eka Suryana, M.Cs dan Med Irzal, M.Cs.
\newline
\newline
\textit{Fish farming in freshwater environments is one of the crucial aspects in the fisheries sector in Indonesia. Recording cultivation data such as feed dosage, fish mortality, fish grading, sorting, and related pond control factors like water conditions and pond data is essential in fish farming. Currently, these data recordings are still done using paper, which poses a risk of errors in calculations. This research aims to develop a supportive application for modern fisheries technology that aids in recording activities within fish farming. This research falls under the category of Research and Development. Information and requirements were obtained through discussions with freshwater fish farmers at JFT (J Farm Technology) and through literature review by studying various journals relevant to the research topic. The discussions resulted in user requirements that serve as a reference for the recording features to be implemented in the modern fisheries technology supportive application. The system development process employs the Scrum methodology, and the entire application is built using the Dart programming language with the Flutter framework. The ultimate outcome of this study is a modern fisheries technology supportive application that assists farmers in recording cultivation activities. Based on the results of the User Acceptance Test (UAT), 6 features were found to be in alignment, while 4 features did not meet field requirements initially. However, these 4 features were subsequently improved to meet the needs of freshwater fish farmers at JFT (J Farm Technology).}
\newline
\newline
\noindent \textbf{Keywordsi}: \textit{android, application, cultivation of fish, fisheries, scrum, recording}
\end{spacing}