\chapter*{\centering{\large{ABSTRAK}}}

\begin{spacing}{1}
\textbf{GIAN CHIESA MAGHRIZA}. Perancangan Aplikasi Pendukung Teknologi Perikanan Modern Berbasis Android. Skripsi. Program Studi Ilmu Komputer Fakultas Matematika dan Ilmu Pengetahuan Alam, Universitas Negeri Jakarta. 2023. Di bawah bimbingan Muhammad Eka Suryana, M.Kom dan Med Irzal, M.Kom.
\newline
\newline
Budidaya ikan di perairan tawar merupakan salah satu aspek penting dalam sektor perikanan di Indonesia. Pencatatan data masa budidaya seperti dosis pakan, kematian ikan, grading ikan, dan sortir ikan serta hal yang berkaitan dengan kontrol kolam seperti kondisi air, data kolam merupakan hal penting dalam budidaya ikan. Saat ini pencatatan data-data tersebut masih dilakukan dengan munggunakan kertas sehingga beresiko akan kesalahan dalam perhitungan. Penelitian ini bertujuan untuk membuat aplikasi pendukung teknologi perikanan modern yang membantu pencatatan dalam melakukan aktivitas budidaya perikanan. Jenis Penelitian ini adalah Pengembangan/\textit{Research} and Development. Informasi dan kebutuhan diperoleh dari diskusi yang dilakukan bersama pembudidaya ikan air tawar di \textit{JFT (J Farm Technology)} serta penelusuran literatur melalui membaca berbagai jurnal yang relevan dengan topik penelitian. Diskusi menghasilkan suatu \textit{user requirement} yang menjadi acuan fitur pencatatan yang akan diterapkan pada aplikasi pendukung teknologi perikanan modern. Proses pengembangan sistem ini menggunakan metode \textit{Scrum} dan seluruh aplikasi yang dibuat menggunakan bahasa pemrograman \textit{Dart} dengan \textit{framework flutter}. Hasil akhir dari penelitian ini berupa aplikasi pendukung teknologi perikanan modern yang membantu pembudidaya dalam melakukan pencatatan aktivitas budidaya. Berdasarkan hasil uji coba \textit{User Acceptance Test (UAT)} terdapat 6 fitur yang sudah sesuai dan 4 fitur yang belum sesuai dengan kebutuhan dilapangan, namun 4 fitur tersebut telah diperbaiki sesuai dengan kebutuhan dari pembudidaya ikan air tawar di \textit{JFT (J Farm Technology)}.
\newline
\newline
\noindent \textbf{Kata kunci}: \textit{android, aplikasi, budidaya ikan, teknologi perikanan, scrum, pencatatan}
\end{spacing}