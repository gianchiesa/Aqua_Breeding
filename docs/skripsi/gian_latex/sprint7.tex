%!TEX root = ./template-skripsi.tex

\subsection{\textit{Sprint 8}}

	\textit{Sprint-8} dilakukan sepekan pada tanggal 4 oktober 2022 sampai dengan 11 oktober 2022. \textit{Story} ketujuh pada \textit{product backlog} yaitu mengintegrasikan fitur treatment kolam dipecah menjadi beberapa \textit{task} sebagai berikut.


 \begin{longtable}[c]{@{} |p{1cm}|p{4cm}|p{5cm}|p{3cm}| @{}}
 \caption{\textit{Sprint 7} \label{sprint7_table}}\\


 \hline
  \multirow{1}{=}{\centering{\textbf{No}}} & \multirow{1}{=}{\centering{\textbf{\textit{Story}}}} & \multirow{1}{=}{\centering{\textbf{\textit{Task}}}} & \multirow{1}{=}{\centering{\textbf{\textit{Status}}}}\\
 \endfirsthead

 \hline
  \multirow{1}{=}{\centering{\textbf{No}}} & \multirow{1}{=}{\centering{\textbf{\textit{Story}}}} & \multirow{1}{=}{\centering{\textbf{\textit{Task}}}} & \multirow{1}{=}{\centering{\textbf{\textit{Status}}}}\\
 \endhead

 \hline
 \endfoot

 \hline
 \endlastfoot

 \hline
 1 & Mengitegrasikan fitur treatment kolam dengan webservice &  Mengitegrasikan fitur treatment kolam dengan webservice &  selesai \\
 \hline
 \end{longtable}

Pada sprint ketujuh ini story yang di pilih untuk di uraikan pada sprint kali ini adalah membuat halaman rekapitulasi treatment kolam, entry treatment kolam. Tujuan dari \textit{sprint-7} ini adalah membuat fitur rekapitulasi treatment kolam dan mengintegrasikan halaman tersebut dengan webservice yang sudah dibuat oleh penelitian Andri Rahmanto.

\begin{enumerate}[listparindent=2em]
	
	\item{\textit{Mengitegrasikan fitur treatment kolam dengan webservice}}

	Sebelumnya pada sprint keenam, setiap data pada fitur treatment masih berupa data dummy sehingga perlu diintegrasikan dengan webservice agar aplikasi dapat berjalan dengan data yang asli. Hal yang dilakukan dalam mengintegrasikan fitur treatment dengan webservice terdapat pada lampiran 8.
	

\item{Sprint 7 Review dan Sprint 8 Planning}

Sprint 7 diakhiri dengan melakukan weekly meeting pada hari selasa dengan agenda melakukan review dan testing terkait hasil sprint 7 dan melakukan planning untuk sprint 8 dengan rincian:
\begin{enumerate}
	\item{\textit{Review dan Testing hasil dari sprint 8}}

	Telah dilakukan review dan testing oleh penulis selaku developer dengan Scrum Master. Setelah dilakukan testing, Scrum Master menyimpulkan bahwa fitur treatment kolam telah berjalan dengan baik.

  \begin{longtable}{| p{8cm} | c | c | l |}
    \caption{Unit testing Halaman Rekapitulasi Treatment.\label{table:unit_testing_rekapitulasi_treatment}}\\
    \hline
    \multirow{2}{*}{Skenario Pengujian} & \multicolumn{2}{l|}{Kesesuaian} & \multirow{2}{*}{Kesimpulan} \\ 
    \cline{2-3}
      & \multicolumn{1}{l|}{sesuai} & tidak sesuai & \\ 
    \hline
    \hline
    \endfirsthead
    \hline
    \multirow{2}{*}{Skenario Pengujian} & \multicolumn{2}{l|}{Kesesuaian} & \multirow{2}{*}{Kesimpulan} \\ 
    \cline{2-3}
      & \multicolumn{1}{l|}{sesuai} & tidak sesuai &  \\ 
    \hline
    \hline
    \endhead
    \hline
    \endfoot
    
    
    \hline\hline
    \endlastfoot
    Ketika menekan list data treatment, maka akan ditamplikan detail treatment & \Checkmark &  & Diterima \\ 
    \hline
    Saat ikon (+) ditekan maka akan menampilkan halaman entry treatment & \Checkmark &  & Diterima \\ 
    \hline
    ketika mengisi form treatment dengan data yang sesuai dan menekan submit, data treatment akan ditambahkan & \Checkmark &  & Diterima \\ 
    \hline
    \end{longtable}    

	\item{\textit{Sprint Planning untuk Sprint 8}}
	
	Planning untuk sprint 8 yakni membuat fitur pencatatan kualitas air harian pada aplikasi \textit{Assistive Aquaculture Breeding Management}.
\end{enumerate}
\end{enumerate}