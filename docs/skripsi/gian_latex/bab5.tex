%!TEX root = ./template-skripsi.tex
%-------------------------------------------------------------------------------
%                            	BAB IV
%               		KESIMPULAN DAN SARAN
%-------------------------------------------------------------------------------

\chapter{KESIMPULAN DAN SARAN}

\section{Kesimpulan}
Berdasarkan hasil implementasi dan pengujian fitur yang telah dirancang, maka diperoleh kesimpulan sebagai berikut:

\begin{enumerate}
	\item Terciptanya aplikasi pendukung teknologi perikanan modern versi pertama yang sudah mengintegrasikan fitur-fitur pada Product Backlog. Adapun perancangannya dilakukan dengan metode Scrum dengan tahapan penyusunan Product Backlog, Sprint Backlog, dan dikerjakan dalam sebelas Sprint.
	
	\item Berdasarkan hasil pengujian, seluruh skenario pada unit testing terhadap satu internal developer berjalan dengan baik. Namun, berdasarkan hasil UAT terhadap pembudidaya terdapat beberapa masukan seperti Pada fitur aktivasi kolam perlu ditambahkan input tipe aktivasi, apakah aktivasi tersebut atau pembesaran , pada fitur kualitas air inputnya perlu bisa dalam format desimal, dan pada fitur sortir, terdapat perubahan flow agar lebih sesuai dengan sortir ikan yang dilakukan oleh pembudidaya.
\end{enumerate}

\section{Saran}
Adapun saran untuk penelitian selanjutnya adalah:
\begin{enumerate} 
	\item Berdasarkan diskusi dengan user/product owner, harus dimulainya pengembangan sistem berikutnya yang memiliki fitur invetarisasi dan penentuan harga jual ikan agar dapat membantu pembudidaya dalam menentukan harga hasil buidayanya.
	\item Berdasarkan diskusi dengan scrum master, perlu ditambahkan feedback dari sistem seperti pop up atau sejenisnya setelah berinteraksi dengan sistem, apakah interaksi yang dilakukan dengan sistem berhasil atau tidak.
\end{enumerate}


% Baris ini digunakan untuk membantu dalam melakukan sitasi
% Karena diapit dengan comment, maka baris ini akan diabaikan
% oleh compiler LaTeX.
\begin{comment}
\bibliography{daftar-pustaka}
\end{comment}