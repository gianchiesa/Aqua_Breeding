\setcounter{section}{0}
\chapter*{\centering{\large{LAMPIRAN}}}

\section{Lampiran 1 Transkrip Percakapan}
\begin{flushleft}
Hari: Selasa
\linebreak
Tanggal: 23 Agustus 2022
\linebreak
P: Penulis
\linebreak
K: Klien (UD Jfarm)
\linebreak
\linebreak
P: Sistem apa yang akan di buat?
\linebreak
K: Kita akan membuat frontend untuk aplikasi pendukung teknologi perikanan modern
\linebreak
P: Apa saja requirement yang diperlukan dalam aplikasi tersebut?
\linebreak
K: Requirementnya adalah penerapan fitur yang sudah ada dalam backend penelitian dari Andri Rahmanto yang berbentuk REST API.
\linebreak
P: Apa saja fitur yang ada dalam backend yang telah dibuat dalam penelitian Andri Rahmanto?
\linebreak
K: Fiturnya adalah pencatatan pemberian pakan, registrasi kolam, aktifasi-deaktifasi kolam, pencatatan kulaitas air harian dan mingguan, pencatatan data kematian harian, pencatatan treatment kolam, grading berat ikan, dan sortir ikan.
\linebreak
\end{flushleft}



\clearpage
\section{Lampiran 2 code untuk sprint 1 report}
\lstset{frame=tb,
	  language=c++,
	  aboveskip=5mm,
	  belowskip=5mm,
	  showstringspaces=false,
	  columns=flexible,
	  basicstyle={\scriptsize\ttfamily},
	  numbers=none,
	  breaklines=true,
	  breakatwhitespace=true,
	  tabsize=3}

	\begin{lstlisting}
	    Widget title() {
	      return Container(
	        margin: EdgeInsets.only(
	          top: defaultMargin,
	          left: defaultMargin,
	          right: defaultMargin,
	        ),
	        child: Text(
	          'Home',
	          style: primaryTextStyle.copyWith(
	            fontSize: 24,
	            fontWeight: semiBold,
	          ),
	        ),
	      );
	    }
	
	    Widget statistic() {
	      return Container(
	          margin: EdgeInsets.only(
	            top: defaultMargin,
	            left: defaultMargin,
	          ),
	          child: Column(
	            crossAxisAlignment: CrossAxisAlignment.center,
	            mainAxisAlignment: MainAxisAlignment.center,
	            children: [
	              Row(
	                mainAxisAlignment: MainAxisAlignment.spaceBetween,
	                crossAxisAlignment: CrossAxisAlignment.center,
	                children: [
	                  Expanded(
	                      flex: 1,
	                      child: StatisticCard(
	                        title: 'Kolam',
	                        value: controller.statistic.value.total_pond,
	                      )),
	                  Expanded(
	                      flex: 1,
	                      child: StatisticCard(
	                        title: 'Kolam Aktif',
	                        value: controller.statistic.value.active_pond,
	                      )),
	                ],
	              ),
	              SizedBox(
	                height: 16,
	              ),
	              Row(
	                mainAxisAlignment: MainAxisAlignment.spaceBetween,
	                crossAxisAlignment: CrossAxisAlignment.center,
	                children: [
	                  Expanded(
	                      flex: 1,
	                      child: StatisticCard(
	                        title: 'Ikan Hidup',
	                        value: controller.statistic.value.fish_live,
	                        unit: 'Ekor',
	                      )),
	                  Expanded(
	                      flex: 1,
	                      child: StatisticCard(
	                        title: 'Ikan Mati',
	                        value: controller.statistic.value.fish_death,
	                        unit: 'Ekor',
	                      )),
	                ],
	              ),
	              SizedBox(
	                height: 16,
	              ),
	              Row(
	                mainAxisAlignment: MainAxisAlignment.spaceBetween,
	                crossAxisAlignment: CrossAxisAlignment.center,
	                children: [
	                  Expanded(
	                      flex: 1,
	                      child: StatisticCard(
	                        title: 'Panen 2022',
	                        value: controller.statistic.value.fish_harvested,
	                        unit: 'Kg',
	                      )),
	                  Expanded(
	                      flex: 1,
	                      child: StatisticCard(
	                        title: 'Total Pakan',
	                        value: controller.statistic.value.total_feed_dose,
	                        unit: 'Kg',
	                      )),
	                ],
	              ),
	            ],
	          ));
	    }
	\end{lstlisting}
	

\textit{Lampiran 3 code widget untuk section statistik ikan}
\begin{lstlisting}
	Widget fishTitle() {
	  return Container(
		margin: EdgeInsets.only(
		  top: defaultSpace,
		  left: defaultMargin,
		  right: defaultMargin,
		),
		child: Column(
		  crossAxisAlignment: CrossAxisAlignment.start,
		  children: [
			Text(
			  'Total Berat Ikan',
			  style: primaryTextStyle.copyWith(
				fontSize: 24,
				fontWeight: semiBold,
			  ),
			),
		  ],
		),
	  );
	}

	Widget fish() {
	  return Container(
		margin: EdgeInsets.only(top: 14),
		child: SingleChildScrollView(
		  scrollDirection: Axis.horizontal,
		  child: Row(
			children: [
			  SizedBox(
				width: defaultMargin,
			  ),
			  Row(children: [
				FishCard(
				  title: "Lele",
				  value: controller.statistic.value.fishes_weight_lele!,
				  image: "assets/lele.png",
				),
				FishCard(
				  title: "Nila Merah",
				  value: controller.statistic.value.fishes_weight_nilamerah!,
				  image: "assets/nilamerah.png",
				),
				FishCard(
				  title: "Nila Hitam",
				  value: controller.statistic.value.fishes_weight_nilahitam!,
				  image: "assets/nilahitam.png",
				),
				FishCard(
				  title: "Mas",
				  value: controller.statistic.value.fishes_weight_mas!,
				  image: "assets/mas.png",
				),
			  ]),
			],
		  ),
		),
	  );
	}
\end{lstlisting}


\textit{code widget untuk section statistik kondisi air}
\begin{lstlisting}
	Widget waterTitle() {
	  return Container(
		margin: EdgeInsets.only(
		  top: defaultSpace,
		  left: defaultMargin,
		  right: defaultMargin,
		),
		child: Column(
		  crossAxisAlignment: CrossAxisAlignment.start,
		  children: [
			Text(
			  'Kualitas Air',
			  style: primaryTextStyle.copyWith(
				fontSize: 24,
				fontWeight: semiBold,
			  ),
			),
		  ],
		),
	  );
	}

	Widget water() {
	  return Container(
		margin: EdgeInsets.only(top: 14),
		child: SingleChildScrollView(
		  scrollDirection: Axis.horizontal,
		  child: Row(
			children: [
			  SizedBox(
				width: defaultMargin,
			  ),
			  Row(children: [
				WaterCard(
				  title: "pH",
				  normal: controller.statistic.value.ph_normal,
				  abnormal: controller.statistic.value.ph_abnormal,
				),
				WaterCard(
				  title: "DO",
				  normal: controller.statistic.value.do_normal,
				  abnormal: controller.statistic.value.do_abnormal,
				),
				WaterCard(
				  title: "Flok",
				  normal: controller.statistic.value.floc_normal,
				  abnormal: controller.statistic.value.floc_abnormal,
				),
			  ]),
			],
		  ),
		),
	  );
	}
\end{lstlisting}



\textit{code widget untuk bagian bottom navigation bar}
\begin{lstlisting}
	bottomNavigationBar: BottomNavigationBar(
            type: BottomNavigationBarType.fixed,
            backgroundColor: backgroundColor3,
            onTap: controller.changeTabIndex,
            currentIndex: controller.tabIndex,
            items: [
              BottomNavigationBarItem(
                icon: Container(
                  margin: EdgeInsets.only(
                    top: 20,
                    bottom: 5,
                  ),
                  child: Image.asset(
                    'assets/home_secondary.png',
                    width: 25,
                    color: controller.tabIndex == 0
                        ? primaryColor
                        : Color(0xff808191),
                  ),
                ),
                label: '',
              ),
              BottomNavigationBarItem(
                icon: Container(
                  margin: EdgeInsets.only(
                    top: 20,
                    bottom: 5,
                  ),
                  child: Image.asset(
                    'assets/pond_secondary.png',
                    width: 25,
                    color: controller.tabIndex == 1
                        ? primaryColor
                        : Color(0xff808191),
                  ),
                ),
                label: '',
              ),
            ],
          ),
	\end{lstlisting}

	


\textit{Mengintegrasikan dengan webservice}

	\textit{Model Class untuk Halaman Dashboard}
	\begin{lstlisting}
	class StatisticModel {
  	int? total_pond;
  	int? active_pond;
  	int? fish_live;
  	int? fish_death;
  	int? fish_harvested;
  	int? total_feed_dose;
  	num? fishes_weight_lele;
  	num? fishes_weight_nilamerah;
  	num? fishes_weight_nilahitam;
  	num? fishes_weight_mas;
  	num? fishes_weight_patin;
  	int? ph_normal;
  	int? ph_abnormal;
  	int? do_normal;
  	int? do_abnormal;
  	int? floc_normal;
	  int? floc_abnormal;

	  StatisticModel(
	      {this.total_pond,
	      this.active_pond,
	      this.fish_live,
	      this.fish_death,
	      this.fish_harvested,
	      this.total_feed_dose,
	      this.fishes_weight_lele,
	      this.fishes_weight_nilamerah,
	      this.fishes_weight_nilahitam,
	      this.fishes_weight_mas,
	      this.fishes_weight_patin,
	      this.ph_normal,
	      this.ph_abnormal,
	      this.do_normal,
	      this.do_abnormal,
	      this.floc_normal,
	      this.floc_abnormal});

	  StatisticModel.fromJson(Map<String, dynamic> json) {
	    total_pond = json['total_pond'];
	    active_pond = json['active_pond'];
	    fish_live = json['fish_live'];
	    fish_death = json['fish_death'];
	    fish_harvested = json['fish_harvested'];
	    total_feed_dose = json['total_feed_dose'];
	    fishes_weight_nilahitam = json['fishes_weight'][0]["amount"];
	    fishes_weight_nilamerah = json['fishes_weight'][1]["amount"];
	    fishes_weight_lele = json['fishes_weight'][2]["amount"];
	    fishes_weight_patin = json['fishes_weight'][3]["amount"];
	    fishes_weight_mas = json['fishes_weight'][4]["amount"];
	    ph_normal = json['water_quality']['ph']['normal'];
	    ph_abnormal = json['water_quality']['ph']['abnormal'];
	    do_normal = json['water_quality']['do']['normal'];
	    do_abnormal = json['water_quality']['do']['abnormal'];
	    floc_normal = json['water_quality']['floc']['normal'];
	    floc_abnormal = json['water_quality']['floc']['abnormal'];
	  }
	}
	\end{lstlisting}

	\textit{ Networt Request untuk Halaman Dashboard }
	\begin{lstlisting}
	import 'dart:convert';
	import 'package:fish/models/statistic_model.dart';
	import 'package:fish/service/url_api.dart';
	import 'package:http/http.dart' as http;
	
	class StatisticService {
	  Future<StatisticModel> getStatistic() async {
	    var url = Uri.parse('http://jft.web.id/fishapi/api/statistic');
	    var headers = {'Content-Type': 'application/json'};
	
	    var response = await http.get(url, headers: headers);
	
	    print(response.body);
	
	    if (response.statusCode == 200) {
	      var data = jsonDecode(response.body);
	      StatisticModel statistic = StatisticModel.fromJson(data);
	
	      return statistic;
	    } else {
	      throw Exception('Gagal Get Products!');
	    }
	  }
	}
	
	\end{lstlisting}

	\textit{ Mengolah data yang telah direquest pada Controller }
	\begin{lstlisting}
	class HomeController extends GetxController {
	  var isLoading = false.obs;
	
	  final statistic = StatisticModel().obs;
	
	  @override
	  void onInit() async {
	    await getStatisticData();
	    super.onInit();
	  }
	
	  Future<void> getStatisticData() async {
	    isLoading.value = true;
	    StatisticModel statisticData = await StatisticService().getStatistic();
	    statistic.value = statisticData;
	    Timer(const Duration(seconds: 1), () {
	      isLoading.value = false;
	    });
	  }
	}

	\end{lstlisting}

	\textit{Menampilkan data pada Halaman Dashboard}
	\begin{lstlisting}
	Widget water() {
	      return Container(
	        margin: EdgeInsets.only(top: 14),
	        child: SingleChildScrollView(
	          scrollDirection: Axis.horizontal,
	          child: Row(
	            children: [
	              SizedBox(
	                width: defaultMargin,
	              ),
	              Row(children: [
	                WaterCard(
	                  title: "pH",
	                  normal: controller.statistic.value.ph_normal,
	                  abnormal: controller.statistic.value.ph_abnormal,
	                ),
	                WaterCard(
	                  title: "DO",
	                  normal: controller.statistic.value.do_normal,
	                  abnormal: controller.statistic.value.do_abnormal,
	                ),
	                WaterCard(
	                  title: "Flok",
	                  normal: controller.statistic.value.floc_normal,
	                  abnormal: controller.statistic.value.floc_abnormal,
	                ),
	              ]),
	            ],
	          ),
	        ),
	      );
	    }
	\end{lstlisting}

\clearpage
\section{Lampiran 3 Sprint 2 report}
\begin{lstlisting}
	import 'package:fish/models/pond_model.dart';
	import 'package:fish/pages/component/pond_card.dart';
	
	import 'package:fish/pages/pond/add_pond_page.dart';
	import 'package:fish/pages/pond/pond_controller.dart';
	import 'package:flutter/material.dart';
	import 'package:fish/theme.dart';
	import 'package:get/get.dart';
	
	class PondPage extends StatefulWidget {
	  PondPage({Key? key}) : super(key: key);
	  @override
	  State<PondPage> createState() => _PondPageState();
	}
	
	class _PondPageState extends State<PondPage> {
	  final PondController controller = Get.put(PondController());
	  int? _value = null;
	  final chip = ["Aktif", "Panen", "Tidak Aktif"];
	  @override
	  void initState() {
		super.initState();
	
		controller.getPondsData(context);
	  }
	
	  @override
	  Widget build(BuildContext context) {
		Widget title() {
		  return Container(
			margin: EdgeInsets.only(
			  top: defaultMargin,
			  left: defaultMargin,
			  right: defaultMargin,
			),
			child: Text(
			  'Kolam',
			  style: primaryTextStyle.copyWith(
				fontSize: 24,
				fontWeight: semiBold,
			  ),
			),
		  );
		}
	
		Widget filter() {
		  return Container(
			margin: EdgeInsets.only(
			  top: defaultMargin,
			  left: defaultMargin,
			  right: defaultMargin,
			),
			child: Wrap(
			  spacing: 8.0,
			  children: List<Widget>.generate(
				3,
				(int index) {
				  return ChoiceChip(
					label: Text(
					  chip[index],
					  style: TextStyle(color: Colors.white),
					),
					shape: StadiumBorder(side: BorderSide(color: Colors.white)),
					selected: _value == index,
					backgroundColor: backgroundColor1,
					selectedColor: primaryColor,
					onSelected: (bool selected) {
					  setState(() {
						_value = selected ? index : null;
						if (_value == null) {
						  controller.getPondsData(context);
						  // return null;
						} else {
						  controller.getPondsFiltered(chip[index]);
						}
					  });
					},
				  );
				},
			  ).toList(),
			),
		  );
		}
	
		Widget pondList() {
		  return Container(
			margin: EdgeInsets.only(top: 14),
			child: SingleChildScrollView(
			  child: ListView.builder(
				shrinkWrap: true,
				primary: false,
				itemBuilder: ((context, index) {
				  return PondCard(pond: controller.ponds[index]
					  // indicatorWater: controller.indicatorWater[index]);
					  );
				}),
				itemCount: controller.ponds.length,
			  ),
			),
		  );
		}
	
		Widget emptyListPond() {
		  return Container(
			  width: double.infinity,
			  margin: EdgeInsets.only(right: defaultMargin, left: defaultMargin),
			  child: Center(
				child: Column(children: [
				  SizedBox(height: 35),
				  Image(
					image: AssetImage("assets/unavailable_icon.png"),
					width: 100,
					height: 100,
					fit: BoxFit.fitWidth,
				  ),
				  SizedBox(height: 20),
				  Text(
					"Anda belum pernah melakukan registrasi kolam",
					style: primaryTextStyle.copyWith(
					  fontSize: 14,
					  fontWeight: bold,
					),
					textAlign: TextAlign.center,
					overflow: TextOverflow.ellipsis,
					maxLines: 2,
				  ),
				  SizedBox(height: 10),
				  Text(
					"Silahkan registrasi kolam",
					style: secondaryTextStyle.copyWith(
					  fontSize: 13,
					  fontWeight: bold,
					),
					textAlign: TextAlign.center,
					overflow: TextOverflow.ellipsis,
					maxLines: 2,
				  ),
				]),
			  ));
		}
	
		return Obx(() {
		  if (controller.isLoading.value == false) {
			return Scaffold(
			  backgroundColor: backgroundColor1,
			  floatingActionButton: FloatingActionButton(
				onPressed: () {
				  Get.to(() => AddPondPage());
				},
				backgroundColor: primaryColor,
				child: const Icon(Icons.add),
			  ),
			  body: ListView(
				children: [
				  title(),
				  filter(),
				  controller.ponds.isEmpty ? emptyListPond() : pondList(),
				  SizedBox(
					height: 10,
				  )
				],
			  ),
			);
		  } else {
			return Center(
			  child: CircularProgressIndicator(
				color: secondaryColor,
			  ),
			);
		  }
		});
	  }
	}
		\end{lstlisting}


\textit{Code widget untuk registrasi kolam}
\begin{lstlisting}
	import 'package:fish/pages/pond/pond_controller.dart';
	import 'package:flutter/material.dart';
	import 'package:fish/theme.dart';
	import 'package:flutter/services.dart';
	import 'package:get/get.dart';
	
	class AddPondPage extends StatefulWidget {
	  const AddPondPage({Key? key}) : super(key: key);
	
	  @override
	  State<AddPondPage> createState() => _AddPondPageState();
	}
	
	class _AddPondPageState extends State<AddPondPage> {
	  final PondController controller = Get.put(PondController());
	
	  @override
	  void dispose() {
		controller.aliasController.clear();
		controller.diameterController.clear();
		controller.heightController.clear();
		controller.locationController.clear();
		controller.lengthController.clear();
		controller.widthController.clear();
		super.dispose();
	  }
	
	  @override
	  Widget build(BuildContext context) {
	  Widget aliasInput() {...}
	  Widget locationInput() {...}
	  Widget materialInput() {...}
	  Widget shapelInput() {...}
	  Widget heightInput() {...}
	  Widget lengthInput() {...}
	  Widget widthInput() {...}
	  Widget diameterInput() {...}
	  Widget registerButton() {...}
	  Widget persegiInput() {...}
	   Widget bundarInput() {...}
	
		return Obx(() {
		  if (controller.isLoading.value == false) {
			return Scaffold(
			  appBar: AppBar(
				backgroundColor: backgroundColor2,
				title: const Text("Registrasi Kolam"),
			  ),
			  backgroundColor: backgroundColor1,
			  body: ListView(
				children: [
				  aliasInput(),
				  locationInput(),
				  materialInput(),
				  shapelInput(),
				  controller.shapeController.selected.value == 'persegi'
					  ? persegiInput()
					  : bundarInput(),
				  heightInput(),
				  registerButton(),
				  SizedBox(
					height: 8,
				  )
				],
			  ),
			);
		  } else {
			return Center(
			  child: CircularProgressIndicator(
				color: secondaryColor,
			  ),
			);
		  }
		});
	  }
	}
	
\end{lstlisting}



\textit{Code widget untuk halaman detail kolam}

\begin{lstlisting}
	import 'dart:developer';

import 'package:fish/models/pond_model.dart';
import 'package:fish/pages/component/activation_card.dart';
import 'package:fish/pages/pond/activation_breed_controller.dart';
import 'package:fish/pages/pond/activation_breed_page.dart';
import 'package:fish/pages/pond/pond_controller.dart';
import 'package:fish/pages/pond/add_pond_page.dart';
import 'package:fish/pages/pond/deactivation_breed_page.dart';
import 'package:fish/pages/pond/detail_pond_controller.dart';
import 'package:flutter/material.dart';
import 'package:fish/theme.dart';
import 'package:get/get.dart';

import '../fish_transfer/fish_transfer_entry_page.dart';

class DetailPondPage extends StatefulWidget {
const DetailPondPage({Key? key}) : super(key: key);

@override
State<DetailPondPage> createState() => _DetailPondPageState();
}

class _DetailPondPageState extends State<DetailPondPage> {
var detailController = Get.put(DetailPondController(), permanent: false);
@override
void initState() {
 super.initState();

 detailController.getPondActivation(context);
}

@override
void activate() {
 print('ini aktif');
 super.activate();
}

@override
void deactivate() {
 print('ini deaktif');
 super.deactivate();
}

@override
Widget build(BuildContext context) {
 Widget pondStatus() {...}

 Widget activationButton() {...}

 Widget deactivationButton() {...}

 Widget detail() {...}

 Widget activationTitle() {...}

 Widget listActivation() {...}

 Widget emptyListActivation() {...}

 return Scaffold(
   backgroundColor: backgroundColor1,
   body: Obx(
	 () => detailController.isLoading.value
		 ? Center(
			 child: CircularProgressIndicator(
			   color: secondaryColor,
			 ),
		   )
		 : ListView(
			 children: [
			   pondStatus(),
			   detailController.isPondActive.value == false
				   ? activationButton()
				   : deactivationButton(),
			   detail(),
			   activationTitle(),
			   detailController.activations.isEmpty
				   ? emptyListActivation()
				   : listActivation(),
			   SizedBox(
				 height: 10,
			   )
			 ],
		   ),
   ),
 );
}
}
 \end{lstlisting}


 \textit{Code halaman aktivasi kolam}

 \begin{lstlisting}
	import 'dart:developer';

import 'package:fish/pages/pond/activation_breed_controller.dart';
import 'package:fish/pages/pond/detail_pond_controller.dart';
import 'package:fish/pages/pond/detail_pond_page.dart';
import 'package:fish/service/pond_service.dart';
import 'package:fish/service/activation_service.dart';
import 'package:flutter/material.dart';
import 'package:fish/theme.dart';
import 'package:flutter/services.dart';
import 'package:get/get.dart';

import '../component/detail_pond_tabview.dart';

class ActivationBreedPage extends StatelessWidget {
  ActivationBreedPage({Key? key}) : super(key: key);

  final ActivationBreedController controller =
      Get.put(ActivationBreedController());

  final DetailPondController detailPondController =
      Get.put(DetailPondController());

  @override
  Widget build(BuildContext context) {
    Widget checkBoxFish() {...}

    Widget waterHeightInput() {...}

    Widget leleInput() {...}

    Widget nilaMerahInput() {...}

    Widget nilaHitamInput() {...}
  
    Widget patinInput() {...}

    Widget masInput() {...}

    Widget breedOptionInput() {...}

    Widget pembesaranInput() {...}

    Widget benihInput() {...}

    Widget activationButton() {...}

    return Obx(() {
      if (controller.isActivationProgress.value == false) {
        return Scaffold(
          appBar: AppBar(
            backgroundColor: backgroundColor2,
            title: const Text("Aktivasi Kolam"),
          ),
          backgroundColor: backgroundColor1,
          body: ListView(
            children: [
              breedOptionInput(),
              controller.breedOptionController.selected.value == "Benih"
                  ? benihInput()
                  : pembesaranInput(),
              checkBoxFish(),
              controller.isNilaHitam == true ? nilaHitamInput() : Container(),
              controller.isNilaMerah == true ? nilaMerahInput() : Container(),
              controller.isLele == true ? leleInput() : Container(),
              controller.isPatin == true ? patinInput() : Container(),
              controller.isMas == true ? masInput() : Container(),
              waterHeightInput(),
              activationButton(),
              SizedBox(
                height: 8,
              )
            ],
          ),
        );
      } else {
        return Center(
          child: CircularProgressIndicator(
            color: secondaryColor,
          ),
        );
      }
    });
  }
}

	\end{lstlisting}

	\clearpage
	\section{Lampiran 4 Code Sprint 3 report}
	\textit{Code untuk model class kolam}
	
	\begin{lstlisting}
	import 'package:flutter/material.dart';
import 'package:intl/intl.dart';

enum PondStatus {
  active,
  nonActive,
  close,
}

class string {}

class Pond {
  String? id;
  int? idInt;
  String? alias;
  String? location;
  String? shape;
  String? material;
  num? length;
  num? width;
  num? diameter;
  num? height;
  num? area;
  num? volume;
  String? buildAt;
  String? imageLink;
  bool? isActive;
  num? fishAlive;
  String? lastActivationDate;
  String? rangeFromLastActivation;
  PondStatus pondStatus;
  String? pondStatusStr;
  String? pondPhDesc;
  num? pondPh;
  String? pondDoDesc;
  num? pondDo;
  num? pondTemp;
  String? status;

  Pond({
    required this.id,
    required this.idInt,
    required this.alias,
    required this.location,
    required this.shape,
    required this.material,
    required this.isActive,
    required this.pondStatus,
    this.length,
    this.width,
    this.diameter,
    this.height,
    this.area,
    this.volume,
    this.buildAt,
    this.imageLink,
    this.fishAlive,
    this.lastActivationDate,
    this.rangeFromLastActivation,
    this.pondStatusStr,
    this.pondPh,
    this.pondPhDesc,
    this.pondDo,
    this.pondDoDesc,
    this.pondTemp,
    this.status,
  });

  factory Pond.fromJson(Map<String, dynamic> json) {
    return Pond(
        id: json['_id'],
        idInt: json['id_int'],
        alias: json['alias'],
        location: json['location'],
        shape: json['shape'],
        material: json['material'],
        length: json['length'],
        width: json['width'],
        diameter: json['diameter'],
        height: json['height'],
        area: json['area'],
        volume: json['volume'],
        buildAt: json['build_at'],
        imageLink: json['image_link'],
        isActive: json['isActive'],
        fishAlive: json['fish_alive'] ?? 0,
        lastActivationDate: json['activation_date'] ?? "-",
        rangeFromLastActivation: json['isActive'] == false
            ? "-"
            : (DateTime.now()
                    .difference(stringToDate(json['activation_date']))
                    .inDays)
                .toString(),
        pondStatus: PondStatusConverter.toEnum(json["status"]),
        pondStatusStr: json["status"],
        pondPh: json["pondPh"],
        pondPhDesc: json["pondPhDesc"],
        pondDo: json["pondDo"],
        pondDoDesc: json["pondDoDesc"],
        pondTemp: json["pondTemp"],
        status: json["status"]);
  }

  static DateTime stringToDate(String dateString) {
    DateTime parseDate = DateFormat("dd-MM-yyyy").parse(dateString);
    return parseDate;
  }

  Color getColor() {
    PondStatus pondStatus = this.pondStatus;
    switch (pondStatus) {
      case PondStatus.active:
        return Colors.green;
      case PondStatus.nonActive:
        return Colors.red.shade300;
      case PondStatus.close:
        return Colors.amber;
      default:
        return Colors.red.shade300;
    }
  }

  String getFishAlive() {
    if (isActive == false) {
      return '-';
    } else {
      return fishAlive.toString();
    }
  }

  String getLastActivationDate() {
    if (isActive == false) {
      return '-';
    } else {
      return lastActivationDate!;
    }
  }

  String getGmtToNormalDate() {
    String stringDate = buildAt!;
    DateTime dateTime = DateFormat("yyyy-MM-dd hh:mm:ss").parse(stringDate);
    String newStringDate = DateFormat("dd-MM-yyyy").format(dateTime);
    return newStringDate;
  }
}

extension PondStatusConverter on PondStatus {
  static PondStatus toEnum(String? status) {
    switch (status) {
      case 'Aktif':
        return PondStatus.active;
      case 'Tidak Aktif':
        return PondStatus.nonActive;
      case 'Panen':
        return PondStatus.close;
      default:
        return PondStatus.nonActive;
    }
  }
}
	\end{lstlisting}

	\textit{Code untuk API service halaman terkait kolam kolam}
	
	\begin{lstlisting}
import 'dart:convert';
import 'dart:math';
import 'package:fish/models/pond_model.dart';
import 'package:fish/service/url_api.dart';
import 'package:flutter/material.dart';
import 'package:http/http.dart' as http;
import 'package:shared_preferences/shared_preferences.dart';

import '../theme.dart';

class PondService {
  Future<List<Pond>> getPonds() async {
    WidgetsFlutterBinding.ensureInitialized();
    SharedPreferences prefs = await SharedPreferences.getInstance();
    String token = prefs.getString('token').toString();
    var url = Uri.parse(Urls.ponds);
    var headers = {
      'Content-Type': 'application/json',
      'Authorization': 'Bearer $token'
    };

    var response = await http.get(url, headers: headers);

    print(response.body);

    if (response.statusCode == 200) {
      var data = jsonDecode(response.body);
      List<Pond> ponds = [];

      for (var item in data) {
        ponds.add(Pond.fromJson(item));
      }

      print(ponds);

      return ponds;
    } else {
      throw Exception(e);
    }
  }

  Future<void> getPondDetail({required String pondId}) async {
    var url = Uri.parse(Urls.pond(pondId));
    var headers = {'Content-Type': 'application/json'};

    var response = await http.get(url, headers: headers);

    print(response.body);

    if (response.statusCode == 200) {
      var data = jsonDecode(response.body);
      // Pond pond = Pond.fromJson(data);
      // print(pond);

      // return pond;
    } else {
      throw Exception('Gagal Get Detial Pond!');
    }
  }

  Future<bool> pondRegister(
      {required String? alias,
      required String? location,
      required String? shape,
      required String? material,
      required String? length,
      required String? width,
      required String? diameter,
      required String? height,
      required String? status,
      required Function doInPost,
      required BuildContext context}) async {
    if (diameter!.isNotEmpty) {
      if (diameter.contains(",")) {
        diameter = diameter.replaceAll(',', '.');
      }
    }
    if (length!.isNotEmpty) {
      if (length.contains(",")) {
        length = length.replaceAll(',', '.');
      }
    }
    if (width!.isNotEmpty) {
      if (width.contains(",")) {
        width = width.replaceAll(',', '.');
      }
    }
    if (height!.isNotEmpty) {
      if (height.contains(",")) {
        height = height.replaceAll(',', '.');
      }
    }
    WidgetsFlutterBinding.ensureInitialized();
    SharedPreferences prefs = await SharedPreferences.getInstance();
    String token = prefs.getString('token').toString();
    final response = await http.post(
      Uri.parse(Urls.ponds),
      headers: {
        "Content-Type": "application/x-www-form-urlencoded",
        'Authorization': 'Bearer $token'
      },
      encoding: Encoding.getByName('utf-8'),
      body: {
        "alias": alias,
        "location": location,
        "shape": shape,
        "material": material,
        "status": status,
        "length": length,
        "width": width,
        "diameter": diameter,
        "height": height,
      },
    );

    if (response.statusCode == 200) {
      doInPost();
      Navigator.pop(context);
      return true;
    } else {
      var res = jsonDecode(response.body);

      showDialog<String>(
          context: context,
          builder: (BuildContext context) => AlertDialog(
                title: const Text('Input Error',
                    style: TextStyle(color: Colors.red)),
                content: Text(
                  '${res["message"]}',
                  style: TextStyle(color: Colors.white),
                ),
                backgroundColor: backgroundColor1,
                shape: RoundedRectangleBorder(
                    borderRadius: BorderRadius.all(Radius.circular(16.0))),
                actions: <Widget>[
                  TextButton(
                    onPressed: () => Navigator.pop(context, 'OK'),
                    child: const Text('OK'),
                  ),
                ],
              ));
      return false;
    }
  }
}
	\end{lstlisting}
	
	\textit{Code widget untuk halaman pemberian pakan}
	\begin{lstlisting}
import 'package:fish/models/feed_chart_model.dart';
import 'package:fish/pages/component/feed_month_card.dart';
import 'package:fish/pages/feeding/feed_controller.dart';
import 'package:fish/pages/pond/detail_pond_controller.dart';
import 'package:fish/pages/pond/pond_controller.dart';
import 'package:flutter/material.dart';
import 'package:fish/pages/feeding/feed_entry_page.dart';
import 'package:fish/theme.dart';
import 'package:get/get.dart';
import 'package:syncfusion_flutter_charts/charts.dart';

class DetailFeedPage extends StatelessWidget {
  const DetailFeedPage({Key? key}) : super(key: key);

  @override
  Widget build(BuildContext context) {
    final FeedController controller = Get.put(FeedController());
    final PondController pondController = Get.find();
    final DetailPondController detailPondController = Get.find();

    Widget chartFeed() {...}

    Widget emptyListPond() {...}

    Widget feedDataRecap() {...}

    Widget entryPakanButton() {...}

    Widget detail() {...}

    Widget recapTitle() {...}

    Widget listMonthFeed() {...}

    return Obx(() {
      if (controller.isLoading.value == false) {
        return Scaffold(
          appBar: AppBar(
            backgroundColor: backgroundColor2,
            title: const Text("Detail Pakan Permusim"),
          ),
          backgroundColor: backgroundColor1,
          body: ListView(
            children: [
              chartFeed(),
              feedDataRecap(),
              // detail(),
              entryPakanButton(),
              recapTitle(),
              // chartRecap(),
              controller.list_feedHistoryMonthly.isEmpty
                  ? emptyListPond()
                  : listMonthFeed(),
              SizedBox(
                height: 10,
              )
            ],
          ),
        );
      } else {
        return Center(
          child: CircularProgressIndicator(
            color: secondaryColor,
          ),
        );
      }
    });
  }
}

	\end{lstlisting}

	\textit{Code untuk halaman entry pakan}
	\begin{lstlisting}
import 'package:fish/pages/feeding/feed_entry_controller.dart';
import 'package:flutter/material.dart';
import 'package:fish/theme.dart';
import 'package:flutter/services.dart';
import 'package:get/get.dart';

import 'feed_controller.dart';

class FeedEntryPage extends StatelessWidget {
  const FeedEntryPage({Key? key}) : super(key: key);

  @override
  Widget build(BuildContext context) {
    final FeedEntryController controller = Get.put(FeedEntryController());
    final FeedController feedcontroller = Get.put(FeedController());

    Widget feedTypeInput() {
      return Container(
        margin: EdgeInsets.only(
            top: defaultSpace, right: defaultMargin, left: defaultMargin),
        child: Column(
          crossAxisAlignment: CrossAxisAlignment.start,
          children: [
            Text(
              'Pilih Pakan',
              style: primaryTextStyle.copyWith(
                fontSize: 16,
                fontWeight: medium,
              ),
            ),
            SizedBox(
              height: 12,
            ),
            Container(
              height: 50,
              padding: EdgeInsets.symmetric(
                horizontal: 16,
              ),
              decoration: BoxDecoration(
                color: backgroundColor2,
                borderRadius: BorderRadius.circular(12),
              ),
              child: Center(
                child: Obx(() => DropdownButtonFormField<String>(
                      onChanged: (newValue) => controller.feedTypeFormController
                          .setSelected(newValue!),
                      value: controller.feedTypeFormController.selected.value,
                      items: controller.feedTypeFormController.listFeedType
                          .map((feedtype) {
                        return DropdownMenuItem<String>(
                          value: feedtype.type,
                          child: Text(
                            feedtype.type.toString(),
                            style: primaryTextStyle,
                          ),
                        );
                      }).toList(),
                      dropdownColor: backgroundColor5,
                      decoration: InputDecoration(border: InputBorder.none),
                    )),
              ),
            ),
          ],
        ),
      );
    }

    Widget feedSatuanInput() {
      return Container(
        margin: EdgeInsets.only(
            top: defaultSpace, right: defaultMargin, left: defaultMargin),
        child: Column(
          crossAxisAlignment: CrossAxisAlignment.start,
          children: [
            Text(
              'Pilih Satuan',
              style: primaryTextStyle.copyWith(
                fontSize: 16,
                fontWeight: medium,
              ),
            ),
            SizedBox(
              height: 12,
            ),
            Container(
              height: 50,
              padding: EdgeInsets.symmetric(
                horizontal: 16,
              ),
              decoration: BoxDecoration(
                color: backgroundColor2,
                borderRadius: BorderRadius.circular(12),
              ),
              child: Center(
                child: Obx(() => DropdownButtonFormField<String>(
                      onChanged: (newValue) => controller.feedSatuanController
                          .setSelected(newValue!),
                      value: controller.feedSatuanController.selected.value,
                      items: controller.feedSatuanController.listSatuan
                          .map((feedtype) {
                        return DropdownMenuItem<String>(
                          value: feedtype,
                          child: Text(
                            feedtype.toString(),
                            style: primaryTextStyle,
                          ),
                        );
                      }).toList(),
                      dropdownColor: backgroundColor5,
                      decoration: InputDecoration(border: InputBorder.none),
                    )),
              ),
            ),
          ],
        ),
      );
    }

    Widget feedDosisInput() {
      return Container(
        margin: EdgeInsets.only(
            top: defaultSpace, right: defaultMargin, left: defaultMargin),
        child: Column(
          crossAxisAlignment: CrossAxisAlignment.start,
          children: [
            Text(
              'Dosis Pakan',
              style: primaryTextStyle.copyWith(
                fontSize: 16,
                fontWeight: medium,
              ),
            ),
            SizedBox(
              height: 12,
            ),
            Container(
              height: 50,
              padding: EdgeInsets.symmetric(
                horizontal: 16,
              ),
              decoration: BoxDecoration(
                color: backgroundColor2,
                borderRadius: BorderRadius.circular(12),
              ),
              child: Center(child: Obx(() {
                return TextFormField(
                  style: primaryTextStyle,
                  inputFormatters: <TextInputFormatter>[
                    FilteringTextInputFormatter.deny(RegExp(r'[-+=*#%/,\s]'))
                  ],
                  keyboardType: TextInputType.number,
                  onChanged: controller.doseChanged,
                  onTap: controller.valdose,
                  controller: controller.feedDosisController,
                  decoration: controller.validatedose.value == true
                      ? controller.dose == ''
                          ? InputDecoration(
                              errorText: 'Dosis tidak boleh kosong',
                              isCollapsed: true)
                          : null
                      : InputDecoration.collapsed(
                          hintText: 'ex: 2.1', hintStyle: subtitleTextStyle),
                );
              })),
            ),
          ],
        ),
      );
    }

    Widget submitButton() {
      return Container(
        height: 50,
        width: double.infinity,
        margin: EdgeInsets.only(
            top: defaultSpace * 3, right: defaultMargin, left: defaultMargin),
        child: TextButton(
          onPressed: () async {
            controller.feedDosisController.text == ""
                ? null
                : Navigator.pop(context);
            controller.postFeedHistory();
            feedcontroller.getChartFeed(
                activation_id: controller.activation.id.toString());
            feedcontroller.getWeeklyRecapFeedHistory(
                activation_id: controller.activation.id.toString());
            controller.postDataLog(controller.fitur);
          },
          style: TextButton.styleFrom(
            backgroundColor: primaryColor,
            shape: RoundedRectangleBorder(
              borderRadius: BorderRadius.circular(12),
            ),
          ),
          child: Text(
            'Submit',
            style: primaryTextStyle.copyWith(
              fontSize: 16,
              fontWeight: medium,
            ),
          ),
        ),
      );
    }

    return Obx(() {
      if (controller.isLoading.value == false) {
        return Scaffold(
          appBar: AppBar(
            backgroundColor: backgroundColor2,
            title: Text("Entry Pakan Kolam ${controller.pond.alias}"),
          ),
          backgroundColor: backgroundColor1,
          body: ListView(
            children: [
              // pondInput(),
              feedTypeInput(),
              feedSatuanInput(),
              feedDosisInput(),
              submitButton(),
              SizedBox(
                height: 8,
              )
            ],
          ),
        );
      } else {
        return Center(
          child: CircularProgressIndicator(
            color: secondaryColor,
          ),
        );
      }
    });
  }
}
	\end{lstlisting}

	\textit{Membuat model class untuk fitur pemberian pakan}
	\begin{lstlisting}
	  import 'package:intl/intl.dart';

class FeedHistoryMonthly {
  DateTime? date;
  num? totalFeedWeight;
  num? totalFeed;

  FeedHistoryMonthly({
    required this.date,
    required this.totalFeedWeight,
    required this.totalFeed,
  });

  factory FeedHistoryMonthly.fromJson(Map<String, dynamic> json) {
    return FeedHistoryMonthly(
      date: DateTime.utc(json['year'], json['_id']),
      totalFeedWeight: json['total_feed'],
      totalFeed: json['total_feedhistory'],
    );
  }

  static List<FeedHistoryMonthly> fromJsonList(List<dynamic> list) {
    List<FeedHistoryMonthly> fishes = [];
    for (var item in list) {
      fishes.add(FeedHistoryMonthly.fromJson(item));
    }
    return fishes;
  }

  String getMonthName() => DateFormat('MMM').format(date!);
  String getMonthNameFull() => DateFormat('MMMM').format(date!);

  String getMonth() => DateFormat('yyyy-MM').format(date!);
}

	\end{lstlisting}

	\textit{Membuat network request untuk fitur pemberian pakan}
	\begin{lstlisting}
	import 'dart:developer';
import 'dart:convert';
import 'package:fish/models/FeedHistoryDaily.dart';
import 'package:fish/models/FeedHistoryHourly.dart';
import 'package:fish/models/FeedHistoryMonthly.dart';
import 'package:fish/models/FeedHistoryWeekly.dart';
import 'package:fish/models/feed_chart_model.dart';
import 'package:fish/service/url_api.dart';
import 'package:http/http.dart' as http;

class FeedHistoryService {
  Future<List<FeedChartData>> getChart({required String activation_id}) async {
    var url = Uri.parse(Urls.feedChartApi(activation_id));
    print(url);
    var headers = {'Content-Type': 'application/json'};

    var response = await http.get(url, headers: headers);

    print(response.body);

    if (response.statusCode == 200) {
      var data = jsonDecode(response.body);
      List<FeedChartData> feedChartData = FeedChartData.fromJsonList(data);
      return feedChartData;
    } else {
      throw Exception('Gagal Get Activation!');
    }
  }

  Future<bool> postFeedHistory({
    required String? pondId,
    required String? feedTypeId,
    required String? feedDose,
  }) async {
    print({
      "pond_id": pondId,
      "feed_type_id": feedTypeId,
      "feed_dose": feedDose,
    });
    final response = await http.post(
      Uri.parse(Urls.feedhistorys),
      headers: {
        "Content-Type": "application/x-www-form-urlencoded",
      },
      encoding: Encoding.getByName('utf-8'),
      body: {
        "pond_id": pondId,
        "feed_type_id": feedTypeId,
        "feed_dose": feedDose,
      },
    );

    if (response.statusCode == 200) {
      print(response.body);
      return true;
    } else {
      print(response.body);
      return false;
    }
  }
}
	\end{lstlisting}

	\clearpage
	\section{Lampiran 5 Code Sprint 4 report}

	\textit{Code halaman rekapitulasi grading}
	\begin{lstlisting}
import 'dart:async';
import 'package:fish/models/grading_chart_model.dart';
import 'package:fish/pages/component/grading_card.dart';
import 'package:fish/pages/grading/grading_controller.dart';
import 'package:flutter/material.dart';
import 'package:fish/pages/grading/grading_constanta_edit_page.dart';
import 'package:fish/pages/grading/grading_entry_page.dart';
import 'package:fish/theme.dart';
import 'package:get/get.dart';
import 'package:syncfusion_flutter_charts/charts.dart';

class GradingPage extends StatelessWidget {
  const GradingPage({Key? key}) : super(key: key);

  @override
  Widget build(BuildContext context) {
    final GradingController controller = Get.put(GradingController());

    Widget chartGrading() {...}

    Widget gradingDataRecap() {...}

    Widget entryGradingButton() {...}

    Widget detail() {...}

    Widget recapTitle() {...}

    Widget emptyListGrading() {...}

    Widget listGrading() {...}

    Widget sizingSec() {...}

    return Obx(() {
      if (controller.isLoading.value == false) {
        print('object');
        return Scaffold(
          appBar: AppBar(
            backgroundColor: backgroundColor2,
            title: const Text("Rekapitulasi Grading"),
          ),
          backgroundColor: backgroundColor1,
          body: ListView(
            children: [
              chartGrading(),
              gradingDataRecap(),
 
              sizingSec(),
              entryGradingButton(),
              recapTitle(),

              controller.list_fishGrading.isEmpty
                  ? emptyListGrading()
                  : listGrading(),
              SizedBox(
                height: 10,
              )
            ],
          ),
        );
      } else {
        return Center(
          child: CircularProgressIndicator(
            color: secondaryColor,
          ),
        );
      }
    });
  }
}

	\end{lstlisting}

	\textit{Code untuk halaman entry grading}
	\begin{lstlisting}
import 'package:fish/pages/grading/grading_entry_controller.dart';
import 'package:flutter/material.dart';
import 'package:fish/theme.dart';
import 'package:flutter/services.dart';
import 'package:get/get.dart';

import 'grading_controller.dart';

class GradingEntryPage extends StatelessWidget {
  const GradingEntryPage({Key? key}) : super(key: key);

  @override
  Widget build(BuildContext context) {
    final GradingEntryController controller = Get.put(GradingEntryController());
    final GradingController gradingcontroller = Get.put(GradingController());

    Widget fishTypelInput() {...}

    Widget sampleAmountInput() {...}

    Widget fishWightInput() {...}

    Widget fishLengthAvgInput() {...}

    Widget undersizeInput() {...}

    Widget oversizeInput() {...}

    Widget normalsizeInput() {...}

    Widget submitButton() {...}

    return Obx(() {
      if (controller.isLoading.value == false) {
        return Scaffold(
          appBar: AppBar(
            backgroundColor: backgroundColor2,
            title: Text("Entry Grading ${controller.pond.alias}"),
          ),
          backgroundColor: backgroundColor1,
          body: ListView(
            children: [
              fishTypelInput(),
              sampleAmountInput(),
              fishWightInput(),
              fishLengthAvgInput(),
              normalsizeInput(),
              undersizeInput(),
              oversizeInput(),
              submitButton(),
              SizedBox(
                height: 8,
              )
            ],
          ),
        );
      } else {
        return Center(
          child: CircularProgressIndicator(
            color: secondaryColor,
          ),
        );
      }
    });
  }
}

	\end{lstlisting}
	\textit{Code untuk halaman detail grading}
	\begin{lstlisting}
	    import 'package:fish/pages/grading/detail_grading_controller.dart';
import 'package:flutter/material.dart';
import 'package:fish/theme.dart';
import 'package:get/get.dart';

class DetailGradingPage extends StatelessWidget {
  const DetailGradingPage({Key? key}) : super(key: key);

  @override
  Widget build(BuildContext context) {
    final GradingDetailController controller =
        Get.put(GradingDetailController());

    Widget gradingDataRecap() {...}

    Widget detail() {...}

    Widget titleRecap() {...}

    Widget dataGrading() {...}

    Widget detailGrading() {...}

    return Obx(() {
      if (controller.isLoading.value == false) {
        return Scaffold(
          appBar: AppBar(
            backgroundColor: backgroundColor2,
            title: const Text("Detail Rekap Grading"),
          ),
          backgroundColor: backgroundColor1,
          body: ListView(
            children: [
              gradingDataRecap(),
              detail(),
              titleRecap(),
              dataGrading(),
              detailGrading(),
              SizedBox(
                height: 10,
              )
            ],
          ),
        );
      } else {
        return Center(
          child: CircularProgressIndicator(
            color: secondaryColor,
          ),
        );
      }
    });
  }
}

	\end{lstlisting}

	\textit{Membuat model class grading berat ikan}
	\begin{lstlisting}
import 'package:intl/intl.dart';

class FishGrading {
  String? id;
  String? fishType;
  num? samplingAmount;
  num? avgFishWeight;
  num? avgFishLong;
  num? normalFish;
  num? oversizeFish;
  num? undersizeFish;
  DateTime? gradingAt;

  FishGrading({
    required this.id,
    required this.fishType,
    required this.samplingAmount,
    required this.avgFishWeight,
    this.avgFishLong,
    this.normalFish,
    this.oversizeFish,
    this.undersizeFish,
    this.gradingAt,
  });

  factory FishGrading.fromJson(Map<String, dynamic> json) {
    print(json);
    return FishGrading(
      id: json['_id'],
      fishType: json['fish_type'],
      samplingAmount: json['sampling_amount'],
      avgFishWeight: json['avg_fish_weight'],
      avgFishLong: json['avg_fish_long'],
      normalFish: json['amount_normal_fish'],
      oversizeFish: json['amount_oversize_fish'],
      undersizeFish: json['amount_undersize_fish'],
      gradingAt: DateTime.tryParse(json['grading_at']),
    );
  }

  static List<FishGrading> fromJsonList(List<dynamic> list) {
    List<FishGrading> fishgradings = [];
    for (var item in list) {
      fishgradings.add(FishGrading.fromJson(item));
    }
    return fishgradings;
  }

  String getDate() => DateFormat('dd-MM-yyyy ').format(gradingAt!);
}
	\end{lstlisting}

	\textit{Membuat network service untuk fitur grading}
	\begin{lstlisting}
	import 'dart:convert';
import 'package:fish/models/fishGrading_model.dart';
import 'package:fish/models/grading_chart_model.dart';
import 'package:fish/service/url_api.dart';
import 'package:http/http.dart' as http;

class FishGradingService {
  Future<List<FishGrading>> fetchFishGradings(
      {required String activationId}) async {
    var url = Uri.parse(Urls.fishGrading(activationId));
    var headers = {'Content-Type': 'application/json'};

    var response = await http.get(url, headers: headers);

    print(response.body);

    if (response.statusCode == 200) {
      var data = jsonDecode(response.body);
      List<FishGrading> fishgradings = FishGrading.fromJsonList(data);
      print("success add fishgradings");
      return fishgradings;
    } else {
      throw Exception('Gagal Get fishgradings!');
    }
  }

  Future<List<GradingChartData>> fetchChartFishGradings(
      {required String activationId}) async {
    var url = Uri.parse(Urls.fishGrading(activationId));
    var headers = {'Content-Type': 'application/json'};

    var response = await http.get(url, headers: headers);

    print(response.body);

    if (response.statusCode == 200) {
      var data = jsonDecode(response.body);
      List<GradingChartData> fishgradings = GradingChartData.fromJsonList(data);
      print("success add fishgradings");
      return fishgradings;
    } else {
      throw Exception('Gagal Get fishgradings!');
    }
  }

  Future<bool> postFishGrading({
    required String? pondId,
    required String? fishType,
    required String? samplingAmount,
    required String? avgFishWeight,
    required String? avgFishLong,
    required String? amountNormal,
    required String? amountOver,
    required String? amountUnder,
  }) async {
    if (avgFishWeight!.isNotEmpty) {
      if (avgFishWeight.contains(",")) {
        avgFishWeight = avgFishWeight.replaceAll(',', '.');
      }
    }
    print({
      "pond_id": pondId.toString(),
      "fish_type": fishType,
      "sampling_amount": samplingAmount,
      "avg_fish_weight": avgFishWeight,
      "avg_fish_long": avgFishLong,
      "amount_normal_fish": amountNormal,
      "amount_oversize_fish": amountOver,
      "amount_undersize_fish": amountUnder,
    });
    final response = await http.post(
      Uri.parse(Urls.fishGradings),
      headers: {
        "Content-Type": "application/x-www-form-urlencoded",
      },
      encoding: Encoding.getByName('utf-8'),
      body: {
        "pond_id": pondId,
        "fish_type": fishType,
        "sampling_amount": samplingAmount,
        "avg_fish_weight": avgFishWeight,
        "avg_fish_long": avgFishLong,
        "amount_normal_fish": amountNormal,
        "amount_oversize_fish": amountOver,
        "amount_undersize_fish": amountUnder,
      },
    );

    if (response.statusCode == 200) {
      print(response.body);
      return true;
    } else {
      print(response.body);
      return false;
    }
  }
}
	\end{lstlisting}


	\clearpage
	\section{Lampiran 6 Code Sprint 5 report}
	\textit{Code halaman rekapitulasi kematian ikan}
	\begin{lstlisting}
import 'package:fish/pages/component/death_card.dart';
import 'package:fish/pages/fish/fish_recap_controller.dart';
import 'package:flutter/material.dart';
import 'package:fish/pages/fish/fish_death_entry_page.dart';
import 'package:fish/theme.dart';
import 'package:get/get.dart';
import 'package:syncfusion_flutter_charts/charts.dart';

import '../../models/fish_live_model.dart';

class FishRecapPage extends StatelessWidget {
  const FishRecapPage({Key? key}) : super(key: key);

  @override
  Widget build(BuildContext context) {
    final FishRecapController controller = Get.put(FishRecapController());

    Widget chartDeath() {...}

    Widget fishDataRecap() {...}

    Widget entryDeathButton() {...}

    Widget emptyListPond() {...}

    Widget detail() {...}

    Widget recapTitle() {...}

    Widget listDeath() {...}

    return Obx(() {
      if (controller.isLoading.value == false) {
        return Scaffold(
          appBar: AppBar(
            backgroundColor: backgroundColor2,
            title: const Text("Rekapitulasi Jumlah Kematian"),
          ),
          backgroundColor: backgroundColor1,
          body: ListView(
            children: [
              chartDeath(),
              fishDataRecap(),
              detail(),
              // sizingSec(),
              entryDeathButton(),
              recapTitle(),
              // chartRecap(),
              controller.list_fishDeath.isEmpty ? emptyListPond() : listDeath(),
              SizedBox(
                height: 10,
              )
            ],
          ),
        );
      } else {
        return Center(
          child: CircularProgressIndicator(
            color: secondaryColor,
          ),
        );
      }
    });
  }
}
	\end{lstlisting}

	\textit{Code untuk halaman entry kematian}
	\begin{lstlisting}
import 'package:fish/pages/fish/fish_death_entry_controller.dart';
import 'package:flutter/material.dart';
import 'package:fish/theme.dart';
import 'package:flutter/services.dart';
import 'package:get/get.dart';

import 'fish_recap_controller.dart';

class FishDeathEntryPage extends StatelessWidget {
  const FishDeathEntryPage({Key? key}) : super(key: key);

  @override
  Widget build(BuildContext context) {
    final FishDeathEntryController controller =
        Get.put(FishDeathEntryController());

    final FishRecapController deathcontroller = Get.put(FishRecapController());
    Widget fishTypeInput() {...}

    Widget fishDeathAmountInput() {...}

    Widget submitButton() {...}

    return Obx(() {
      if (controller.isLoading.value == false) {
        return Scaffold(
          appBar: AppBar(
            backgroundColor: backgroundColor2,
            title: Text("Entry Kematian Ikan ${controller.pond.alias}"),
          ),
          backgroundColor: backgroundColor1,
          body: ListView(
            children: [
              fishTypeInput(),
              fishDeathAmountInput(),
              submitButton(),
              SizedBox(
                height: 8,
              )
            ],
          ),
        );
      } else {
        return Center(
          child: CircularProgressIndicator(
            color: secondaryColor,
          ),
        );
      }
    });
  }
}
	\end{lstlisting}

	\textit{Membuat model class rekapitulasi kematian ikan}
	\begin{lstlisting}
import 'package:intl/intl.dart';

class FishDeath {
  String? id;
  String? diagnosis;
  String? fishType;
  num? deathAmount;
  DateTime? deathAt;

  FishDeath({
    required this.id,
    required this.diagnosis,
    required this.fishType,
    required this.deathAmount,
    required this.deathAt,
  });

  factory FishDeath.fromJson(Map<String, dynamic> json) {
    print(json);
    return FishDeath(
      id: json['_id'],
      diagnosis: json['diagnosis'],
      fishType: json['fish']['fish_type'],
      deathAmount: json["fish"]['fish_amount'] * -1,
      deathAt: DateTime.tryParse(json['death_at']),
    );
  }

  static List<FishDeath> fromJsonList(List<dynamic> list) {
    List<FishDeath> fishDeaths = [];
    for (var item in list) {
      fishDeaths.add(FishDeath.fromJson(item));
    }
    return fishDeaths;
  }

  String getDate() => DateFormat('dd-MM-yyyy ').format(deathAt!);
}
	\end{lstlisting}

	\textit{Membuat network service untuk fitur rekapitulasi kematian ikan}
	\begin{lstlisting}
import 'dart:convert';
import 'package:fish/models/fishDeath_model.dart';
import 'package:fish/models/fish_live_model.dart';
import 'package:fish/service/url_api.dart';
import 'package:http/http.dart' as http;

class FishDeathService {
  Future<List<FishDeath>> fetchFishDeaths(
      {required String activationId}) async {
    var url = Uri.parse(Urls.fishDeath(activationId));
    var headers = {'Content-Type': 'application/json'};

    var response = await http.get(url, headers: headers);

    print(response.body);

    if (response.statusCode == 200) {
      var data = jsonDecode(response.body);
      List<FishDeath> fishlive = FishDeath.fromJsonList(data);
      print("success add fishlvie");
      return fishlive;
    } else {
      throw Exception('Gagal Get fishdeath!');
    }
  }

  Future<List<FishLiveData>> fetchFishLive(
      {required String activationId}) async {
    var url = Uri.parse(Urls.fishDeath(activationId));
    var headers = {'Content-Type': 'application/json'};

    var response = await http.get(url, headers: headers);

    print(response.body);

    if (response.statusCode == 200) {
      var data = jsonDecode(response.body);
      List<FishLiveData> fishdeath = FishLiveData.fromJsonList(data);
      print("success add fishdeath");
      return fishdeath;
    } else {
      throw Exception('Gagal Get fishdeath!');
    }
  }

  Future<bool> postFishDeath({
    required String? pondId,
    required List fish,
  }) async {
    print({"pond_id": pondId, "fish_death_amount": fish});
    final response = await http.post(
      Uri.parse(Urls.fishDeaths),
      headers: {
        "Content-Type": "application/x-www-form-urlencoded",
      },
      encoding: Encoding.getByName('utf-8'),
      body: {
        "pond_id": pondId,
        "fish_death_amount": fish.toString(),
        "diagnosis": "mati karena sakit"
      },
    );

    if (response.statusCode == 200) {
      print(response.body);
      return true;
    } else {
      print(response.body);
      return false;
    }
  }
}
	\end{lstlisting}

	
	\clearpage
	\section{Lampiran 7 Code Sprint 6 report}
	\textit{Code halaman rekapitulasi treatment}
	\begin{lstlisting}
    import 'package:fish/models/fishDeath_model.dart';
    import 'package:fish/pages/component/treatment_card.dart';
    import 'package:fish/pages/treatment/treatment_controller.dart';
    import 'package:fish/pages/treatment/treatment_entry_page.dart';
    import 'package:flutter/material.dart';
    import 'package:fish/theme.dart';
    import 'package:get/get.dart';
    
    class TreatmentpPage extends StatefulWidget {
      TreatmentpPage({Key? key}) : super(key: key);
    
      @override
      State<TreatmentpPage> createState() => _TreatmentPageState();
    }
    
    class _TreatmentPageState extends State<TreatmentpPage> {
      final TreatmentController controller = Get.put(TreatmentController());
    
      @override
      void initState() {
        super.initState();
        // WidgetsBinding.instance.addPostFrameCallback((timeStamp) async {
        //   await controller.getPondActivations(
        //       pondId: controller.pond.id.toString());
        // });
        controller.getTreatmentData(context);
      }
    
      @override
      void dispose() {
        controller.postDataLog(controller.fitur);
        super.dispose();
      }
    
      @override
      Widget build(BuildContext context) {...}
    
        Widget listTreatment() {...}
    
        Widget emptyListTreatment() {...}
    
        return Obx(() {
          if (controller.isLoading.value == false) {
            return Scaffold(
              floatingActionButton: FloatingActionButton(
                onPressed: () {
                  Get.to(() => TreatmentEntryPage(), arguments: {
                    "pond": controller.pond,
                    "activation": controller.activation
                  });
                  controller.postDataLog(controller.fitur);
                },
                backgroundColor: primaryColor,
                child: const Icon(Icons.add),
              ),
              backgroundColor: backgroundColor1,
              body: ListView(
                children: [
                  fishDataRecap(),
                  controller.listTreatmentTest.isEmpty
                      ? emptyListTreatment()
                      : listTreatment(),
                  SizedBox(
                    height: 10,
                  )
                ],
              ),
            );
          } else {
            return Center(
              child: CircularProgressIndicator(
                color: secondaryColor,
              ),
            );
          }
        });
    }   
	\end{lstlisting}
	
	\textit{Code untuk halaman entry treatment}
	\begin{lstlisting}
    import 'package:fish/models/fish_model.dart';
    import 'package:fish/pages/component/treatment_berat_input_card.dart';
    import 'package:fish/pages/treatment/treatment_entry_controller.dart';
    import 'package:fish/pages/treatment/treatment_controller.dart';
    import 'package:flutter/material.dart';
    import 'package:fish/pages/pond/detail_pond_controller.dart';
    import 'package:fish/theme.dart';
    
    import 'package:fish/pages/component/deactivation_list_input.dart';
    import 'package:flutter/services.dart';
    import 'package:get/get.dart';
    
    class TreatmentEntryPage extends StatefulWidget {
      TreatmentEntryPage({Key? key}) : super(key: key);
      @override
      State<TreatmentEntryPage> createState() => _TreatmentEntryPageState();
    }
    
    class _TreatmentEntryPageState extends State<TreatmentEntryPage> {
      final TreatmentEntryController controller =
          Get.put(TreatmentEntryController());
    
      final TreatmentController treatmentTontroller =
          Get.put(TreatmentController());
      void initState() {
        super.initState();
        // WidgetsBinding.instance.addPostFrameCallback((timeStamp) async {
        //   await controller.getPondActivations(
        //       pondId: controller.pond.id.toString());
        // });
        controller.getHarvestedBool(controller.activation);
      }
    
      @override
      void dispose() {
        controller.postDataLog(controller.fitur);
        controller.carbonController.clear();
        controller.descController.clear();
        controller.leleWeightController.clear();
        controller.masWeightController.clear();
        controller.nilaHitamWeightController.clear();
        controller.nilaMerahWeightController.clear();
        controller.patinWeightController.clear();
        controller.saltController.clear();
        controller.waterController.clear();
        controller.probioticController.clear();
        super.dispose();
      }
    
      @override
      Widget build(BuildContext context) {
        Widget descInput() {...}
    
        Widget carbonTypeNullInput() {...}
    
        Widget waterChangeInput() {...}
    
        Widget listTreatmentBeratInput() {...}
    
        Widget probioticInput() {...}
    
        Widget carbonInput() {...}
    
        Widget carbonTypeInput() {...}
    
        Widget submitButton() {...}
    
        Widget submitBeratButton() {...}
    
        return Obx(() {
          if (controller.isLoading.value == false) {
            return Scaffold(
              appBar: AppBar(
                backgroundColor: backgroundColor2,
                title: Text("Entry Treatment Kolam ${controller.pond.alias}"),
              ),
              backgroundColor: backgroundColor1,
              body: ListView(
                children: [
                  treatmentTypeInput(),
                  descInput(),
                  controller.typeController.selected.value == "berat"
                      ? listTreatmentBeratInput()
                      : waterChangeInput(),
                  controller.typeController.selected.value == "berat"
                      ? Container()
                      : probioticInput(),
                  controller.typeController.selected.value == "berat"
                      ? Container()
                      : carbonTypeInput(),
                  controller.carbonTypeController.selected.value == "tidak ada"
                      ? Container()
                      : carbonInput(),
                  controller.typeController.selected.value == "berat"
                      ? Container()
                      : saltDosisInput(),
                  controller.typeController.selected.value == "berat"
                      ? submitBeratButton()
                      : submitButton(),
                  SizedBox(
                    height: 8,
                  )
                ],
              ),
            );
          } else {
            return Center(
              child: CircularProgressIndicator(
                color: secondaryColor,
              ),
            );
          }
        });
      }
    }

	\end{lstlisting}


	\textit{Code untuk halaman detail treatment}
	\begin{lstlisting}
    import 'package:fish/pages/treatment/treatment_detail_controller.dart';
    import 'package:flutter/material.dart';
    import 'package:fish/theme.dart';
    import 'package:get/get.dart';
    
    class DetailTreatmentPage extends StatelessWidget {
      const DetailTreatmentPage({Key? key}) : super(key: key);
    
      @override
      Widget build(BuildContext context) {
        final TreatmentDetailController controller =
            Get.put(TreatmentDetailController());
    
        Widget treatmentDataRecap() {...}
    
        Widget detail() {...}
    
        Widget titleRecap() {...}
    
        Widget dataTreatment() {...}
    
        Widget detailTreatment() {...}
    
        return Obx(() {
          if (controller.isLoading.value == false) {
            return Scaffold(
              appBar: AppBar(
                backgroundColor: backgroundColor2,
                title: const Text("Detail Treatment "),
              ),
              backgroundColor: backgroundColor1,
              body: ListView(
                children: [
                  treatmentDataRecap(),
                  detail(),
                  titleRecap(),
                  dataTreatment(),
                  detailTreatment(),
                  SizedBox(
                    height: 10,
                  )
                ],
              ),
            );
          } else {
            return Center(
              child: CircularProgressIndicator(
                color: secondaryColor,
              ),
            );
          }
        });
      }
    }    
	\end{lstlisting}
	\clearpage
	\section{Lampiran 8 Code Sprint 7 report}
	\textit{Membuat model class fitur treatment kolam}
	\begin{lstlisting}
        import 'dart:ffi';

        import 'package:intl/intl.dart';
        
        class Treatment {
          String? id;
          String? activation_id;
          num? salt;
          String? type;
          num? probiotic;
          num? water;
          num? carbohydrate;
          String? carbohydrate_type;
          String? desc;
          String? treatmentAt;
        
          Treatment(
              {this.id,
              this.salt,
              this.type,
              this.probiotic,
              this.water,
              this.carbohydrate,
              this.desc,
              this.carbohydrate_type,
              this.activation_id,
              this.treatmentAt});
        
          // Treatment.fromJson(Map<String, dynamic> json) {
          //   salt = json['salt'];
          //   type = json['treatment_type'];
          //   probiotic = json['probiotic_culture'];
          // }
          factory Treatment.fromJson(Map<String, dynamic> json) {
            return Treatment(
                id: json['id'],
                salt: json['salt'],
                type: json['treatment_type'],
                probiotic: json['probiotic_culture'],
                water: json['water_change'],
                desc: json['description'],
                carbohydrate: json['carbohydrate'],
                carbohydrate_type: json['carbohydrate_type'],
                activation_id: json['pond_activation_id'],
                treatmentAt: json["treatment_at"]);
          }
          String getGmtToNormalDate() {
            String stringDate = treatmentAt!;
            DateTime dateTime = DateFormat("yyyy-MM-dd hh:mm:ss").parse(stringDate);
            String newStringDate = DateFormat("dd-MM-yyyy").format(dateTime);
            return newStringDate;
          }
        }
        
	\end{lstlisting}
	
	\textit{Membuat network service untuk fitur treatment kolam}
	\begin{lstlisting}
        import 'dart:convert';
        import 'package:fish/models/treatment_model.dart';
        import 'package:fish/service/url_api.dart';
        import 'package:http/http.dart' as http;
        
        class TreatmentService {
          Future<List<Treatment>> getTreatmentList() async {
            var url = Uri.parse(Urls.treatment);
            var headers = {'Content-Type': 'application/json'};
        
            var response = await http.get(url, headers: headers);
        
            print(response.body);
        
            if (response.statusCode == 200) {
              var data = jsonDecode(response.body);
              List<Treatment> treatments = [];
        
              for (var item in data) {
                treatments.add(Treatment.fromJson(item));
              }
              // Treatment treatment = Treatment.fromJson(data[0]);
              // print(data[1]);
              return treatments;
            } else {
              throw Exception('Gagal Get Products!');
            }
          }
        
          Future<bool> postPondTreatment({
            required String? pondId,
            String? salt,
            String? type,
            String? probiotic,
            String? water,
            String? desc,
            String? carbohydrate,
            String? carbohydrate_type,
          }) async {
            print({
              "pond_id": pondId.toString(),
              "salt": salt,
              "treatment_type": type,
              "probiotic_culture": probiotic,
              "water_change": water,
              "description": desc,
              "carbohydrate": carbohydrate,
              "carbohydrate_type": carbohydrate_type,
            });
            final response = await http.post(
              Uri.parse(Urls.treatment),
              headers: {
                "Content-Type": "application/x-www-form-urlencoded",
              },
              encoding: Encoding.getByName('utf-8'),
              body: {
                "pond_id": pondId,
                "salt": salt,
                "treatment_type": type,
                "probiotic_culture": probiotic,
                "water_change": water,
                "description": desc.toString(),
                "carbohydrate": carbohydrate,
                "carbohydrate_type": carbohydrate_type,
              },
            );
        
            if (response.statusCode == 200) {
              print(response.body);
              return true;
            } else {
              print(response.body);
              return false;
            }
          }
        
          Future<bool> postPondTreatmentBerat(
              {required String? pondId,
              String? type,
              String? desc,
              required num? total_fish_harvested,
              required num? total_weight_harvested,
              List? fish_harvested,
              bool? isFinish}) async {
            print({
              "pond_id": pondId.toString(),
              "treatment_type": type,
              "description": desc,
            });
            final response = await http.post(
              Uri.parse(Urls.treatment),
              headers: {
                "Content-Type": "application/x-www-form-urlencoded",
              },
              encoding: Encoding.getByName('utf-8'),
              body: {
                "pond_id": pondId,
                "treatment_type": type,
                "description": desc.toString(),
                "total_weight_harvested": total_weight_harvested.toString(),
                "total_fish_harvested": total_fish_harvested.toString(),
                "fish": fish_harvested.toString()
              },
            );
        
            if (response.statusCode == 200) {
              print(response.body);
              return true;
            } else {
              print(response.body);
              return false;
            }
          }
        }        
	\end{lstlisting}

	\clearpage
	\section{Lampiran 9 Code Sprint 8 report}

	\textit{Code halaman list pencatatan kualitas air harian}
	\begin{lstlisting}
import 'package:fish/controllers/daily_water/daily_water_breed_list_controller.dart';
import 'package:fish/pages/component/daily_water_card.dart';
import 'package:fish/controllers/daily_water/daily_water_controller.dart';
import 'package:fish/pages/dailywater/daily_water_entry_page.dart';
import 'package:fish/pages/pond/pond_controller.dart';
import 'package:flutter/material.dart';
import 'package:fish/theme.dart';
import 'package:get/get.dart';
import 'package:http/http.dart';

import 'daily_water_avg.dart';

class DailyWaterPage extends StatefulWidget {
    DailyWaterPage({Key? key}) : super(key: key);

    @override
    State<DailyWaterPage> createState() => _DailyWaterPageState();
}

class _DailyWaterPageState extends State<DailyWaterPage> {
    final DailyWaterController controller = Get.put(DailyWaterController());
    final PondController pondController = Get.find();
    final DailyWaterBreedListController dailyWaterBreedListController =
        Get.find();

    @override
    void initState() {
    super.initState();
    // WidgetsBinding.instance.addPostFrameCallback((timeStamp) async {
    //   await controller.getPondActivations(
    //       pondId: controller.pond.id.toString());
    // });
    controller.getDailyWaterData(context);
    controller.startTime = DateTime.now();
    print('ini init state');
    }

    @override
    void dispose() {
    controller.postDataLog(controller.fitur);
    super.dispose();
    }

    @override
    Widget build(BuildContext context) {
    controller.startTime = DateTime.now();
    print('ini build daily water');
    Widget fishDataRecap() {...}

    Widget listDailyWater() {...}

    Widget emptyList() {...}

    return Obx(() {
        if (controller.isLoading.value == false) {
        return Scaffold(
            floatingActionButton: FloatingActionButton(
            onPressed: () {
                Get.to(() => DailyWaterEntryPage(), arguments: {
                "pond": pondController.selectedPond.value,
                "activation":
                    dailyWaterBreedListController.selectedActivation.value
                });
                controller.postDataLog(controller.fitur);
            },
            backgroundColor: primaryColor,
            child: const Icon(Icons.add),
            ),
            backgroundColor: backgroundColor1,
            body: ListView(
            children: [
                fishDataRecap(),
                controller.listDailyWater.isEmpty
                    ? emptyList()
                    : listDailyWater(),
                SizedBox(
                height: 10,
                )
            ],
            ),
        );
        } else {
        return Center(
            child: CircularProgressIndicator(
            color: secondaryColor,
            ),
        );
        }
    });
    }
}        
	\end{lstlisting}
	\textit{Code untuk halaman entry kualitas air harian}
	\begin{lstlisting}
import 'package:fish/controllers/daily_water/daily_water_entry_controller.dart';
import 'package:fish/controllers/daily_water/daily_water_controller.dart';
import 'package:flutter/material.dart';
import 'package:fish/theme.dart';
import 'package:flutter/services.dart';
import 'package:get/get.dart';

import '../component/tabviewwater.dart';

class DailyWaterEntryPage extends StatelessWidget {
    DailyWaterEntryPage({Key? key}) : super(key: key);

    final DailyWaterEntryController controller =
        Get.put(DailyWaterEntryController());

    final DailyWaterController water = Get.put(DailyWaterController());

    @override
    Widget build(BuildContext context) {
    Widget doInput() {...}

    Widget phInput() {...}

    Widget temperatureInput() {...}

    Widget submitButton() {...}

    return Obx(() {
        if (controller.isLoading.value == false) {
        return Scaffold(
            appBar: AppBar(
            backgroundColor: backgroundColor2,
            title:
                Text("Entry Kondisi Air Harian Kolam ${controller.pond.alias}"),
            ),
            backgroundColor: backgroundColor1,
            body: ListView(
            children: [
                phInput(),
                doInput(),
                temperatureInput(),
                submitButton(),
                SizedBox(
                height: 8,
                )
            ],
            ),
        );
        } else {
        return Center(
            child: CircularProgressIndicator(
            color: secondaryColor,
            ),
        );
        }
    });
    }
}        
	\end{lstlisting}

	\textit{Code untuk halaman detail kualitas air harian}
	\begin{lstlisting}
        import 'package:fish/controllers/daily_water/daily_water_breed_list_controller.dart';
        import 'package:fish/controllers/daily_water/daily_water_detail_controller.dart';
        import 'package:fish/pages/dailywater/daily_water_edit_page.dart';
        import 'package:fish/pages/pond/detail_pond_controller.dart';
        import 'package:fish/pages/pond/pond_controller.dart';
        import 'package:flutter/material.dart';
        import 'package:fish/theme.dart';
        import 'package:flutter/services.dart';
        import 'package:get/get.dart';
        
        import '../../controllers/daily_water/daily_water_controller.dart';
        import '../../controllers/daily_water/daily_water_edit_controller.dart';
        import '../../models/daily_water_model.dart';
        
        class DailyWaterDetailPage extends StatefulWidget {
          const DailyWaterDetailPage({Key? key}) : super(key: key);
        
          @override
          State<DailyWaterDetailPage> createState() => _DailyWaterDetailPageState();
        }
        
        @override
        class _DailyWaterDetailPageState extends State<DailyWaterDetailPage> {
          final DailyWaterDetailController controller =
              Get.put(DailyWaterDetailController());
          final PondController pondController = Get.find();
          final DailyWaterBreedListController dailyWaterBreedListController =
              Get.find();
          final DailyWaterController dailyWaterController = Get.find();
          final DailyWaterEditController editController =
              Get.put(DailyWaterEditController());
          @override
          void initState() {
            super.initState();
            controller.getDailyWaterData(
                context, dailyWaterController.selectedDailyWater.value.id.toString());
          }
        
          @override
          void dispose() {
            controller.postDataLog(controller.fitur);
            dailyWaterController.getDailyWaterData(context);
            super.dispose();
          }
        
          @override
          Widget build(BuildContext context) {
            Widget airHarianDataRecap() {...}
        
            Widget detail() {...}
        
            Widget titleRecap() {...}
        
            Widget dataAirHarian() {...}
        
            Widget detailAirHarian() {...}
        
            return Obx(() {
              if (controller.isLoading.value == false) {
                return Scaffold(
                  appBar: AppBar(
                    backgroundColor: backgroundColor2,
                    title: const Text("Detail Kondisi Air Harian"),
                    leading: new IconButton(
                      icon: new Icon(Icons.arrow_back),
                      onPressed: () async {
                        // Get.back();
        
                        Navigator.pop(context);
                      },
                    ),
                  ),
                  backgroundColor: backgroundColor1,
                  body: ListView(
                    children: [
                      airHarianDataRecap(),
                      detail(),
                      titleRecap(),
                      dataAirHarian(),
                      detailAirHarian(),
                      SizedBox(
                        height: 10,
                      )
                    ],
                  ),
                );
              } else {
                return Center(
                  child: CircularProgressIndicator(
                    color: secondaryColor,
                  ),
                );
              }
            });
          }
        
          void editDataPh(DailyWater dailywater, String title) {
            showDialog<String>(
                context: context,
                builder: (BuildContext context) => AlertDialog(
                      title: Text('Edit $title', style: TextStyle(color: Colors.white)),
                      content: Container(
                        height: 50,
                        padding: EdgeInsets.symmetric(
                          horizontal: 16,
                        ),
                        decoration: BoxDecoration(
                          color: backgroundColor2,
                          borderRadius: BorderRadius.circular(12),
                        ),
                        child: Center(
                            child: TextFormField(
                          style: primaryTextStyle,
                          inputFormatters: <TextInputFormatter>[
                            FilteringTextInputFormatter.deny(RegExp(r'[-+=*#%/,\s]'))
                          ],
                          keyboardType: TextInputType.number,
                          controller: controller.phController,
                        )),
                      ),
                      backgroundColor: backgroundColor1,
                      actions: <Widget>[
                        TextButton(
                          onPressed: () => Navigator.pop(context, 'Batal'),
                          child: const Text(
                            'Batal',
                            style: TextStyle(color: Colors.red),
                          ),
                        ),
                        TextButton(
                          onPressed: () async {
                            await editController.editDailyWaterDataOne(context, () {
                              Navigator.pop(context, 'Submit');
                            },
                                controller.phController.text,
                                dailywater.numDo.toString(),
                                dailywater.temperature.toString());
                            controller.getDailyWaterData(
                                context,
                                dailyWaterController.selectedDailyWater.value.id
                                    .toString());
                          },
                          child: const Text('Submit'),
                        ),
                      ],
                    ));
          }
        
          void editDataSuhu(DailyWater dailywater, String title) {
            showDialog<String>(
                context: context,
                builder: (BuildContext context) => AlertDialog(
                      title: Text('Edit $title', style: TextStyle(color: Colors.white)),
                      content: Container(
                        height: 50,
                        padding: EdgeInsets.symmetric(
                          horizontal: 16,
                        ),
                        decoration: BoxDecoration(
                          color: backgroundColor2,
                          borderRadius: BorderRadius.circular(12),
                        ),
                        child: Center(
                            child: TextFormField(
                          style: primaryTextStyle,
                          inputFormatters: <TextInputFormatter>[
                            FilteringTextInputFormatter.deny(RegExp(r'[-+=*#%/,\s]'))
                          ],
                          keyboardType: TextInputType.number,
                          controller: controller.suhuController,
                        )),
                      ),
                      backgroundColor: backgroundColor1,
                      actions: <Widget>[
                        TextButton(
                          onPressed: () => Navigator.pop(context, 'Batal'),
                          child: const Text(
                            'Batal',
                            style: TextStyle(color: Colors.red),
                          ),
                        ),
                        TextButton(
                          onPressed: () async {
                            await editController.editDailyWaterDataOne(context, () {
                              Navigator.pop(context, 'Submit');
                            }, dailywater.ph.toString(), dailywater.numDo.toString(),
                                controller.suhuController.text);
                            controller.getDailyWaterData(
                                context,
                                dailyWaterController.selectedDailyWater.value.id
                                    .toString());
                          },
                          child: const Text('Submit'),
                        ),
                      ],
                    ));
          }
        
          void editDataDo(DailyWater dailywater, String title) {
            showDialog<String>(
                context: context,
                builder: (BuildContext context) => AlertDialog(
                      title: Text('Edit $title', style: TextStyle(color: Colors.white)),
                      content: Container(
                        height: 50,
                        padding: EdgeInsets.symmetric(
                          horizontal: 16,
                        ),
                        decoration: BoxDecoration(
                          color: backgroundColor2,
                          borderRadius: BorderRadius.circular(12),
                        ),
                        child: Center(
                            child: TextFormField(
                          style: primaryTextStyle,
                          inputFormatters: <TextInputFormatter>[
                            FilteringTextInputFormatter.deny(RegExp(r'[-+=*#%/,\s]'))
                          ],
                          keyboardType: TextInputType.number,
                          controller: controller.doController,
                        )),
                      ),
                      backgroundColor: backgroundColor1,
                      actions: <Widget>[
                        TextButton(
                          onPressed: () => Navigator.pop(context, 'Batal'),
                          child: const Text(
                            'Batal',
                            style: TextStyle(color: Colors.red),
                          ),
                        ),
                        TextButton(
                          onPressed: () async {
                            await editController.editDailyWaterDataOne(context, () {
                              Navigator.pop(context, 'Submit');
                            }, dailywater.ph.toString(), controller.doController.text,
                                dailywater.numDo.toString());
                            controller.getDailyWaterData(
                                context,
                                dailyWaterController.selectedDailyWater.value.id
                                    .toString());
                          },
                          child: const Text('Submit'),
                        ),
                      ],
                    ));
          }
        }        
	\end{lstlisting}

	\textit{Membuat model class fitur pencatatan kualitas air harian}
	\begin{lstlisting}
import 'package:flutter/material.dart';
import 'package:intl/intl.dart';


class DailyWater {
    String? id;
    String? pondId;
    String? activationId;
    num? ph;
    num? numDo;
    num? temperature;
    num? week;
    String? ph_desc;
    String? numDo_desc;
    String? dailywater_at;

    DailyWater(
        {required this.id,
        this.pondId,
        this.activationId,
        this.numDo,
        this.numDo_desc,
        this.temperature,
        this.week,
        this.ph,
        this.ph_desc,
        this.dailywater_at});

    factory DailyWater.fromJson(Map<String, dynamic> json) {
    return DailyWater(
        id: json['_id'],
        pondId: json['pond_id'],
        activationId: json['pond_activation_id'],
        ph: json['ph'],
        numDo: json['do'],
        temperature: json['temperature'],
        week: json['week'],
        ph_desc: json['ph_desc'],
        numDo_desc: json['do_desc'],
        dailywater_at: json['dailywater_at']);
    }
    String getGmtToNormalDate() {
    String stringDate = dailywater_at!;
    DateTime dateTime = DateFormat("yyyy-MM-dd hh:mm:ss").parse(stringDate);
    String newStringDate = DateFormat("dd-MM-yyyy").format(dateTime);
    return newStringDate;
    }

    String getDayNameDate() {
    String stringDate = dailywater_at!;
    DateTime dateTime = DateFormat("yyyy-MM-dd hh:mm:ss").parse(stringDate);
    String newStringDate = DateFormat("EEEE").format(dateTime);
    return newStringDate;
    }

}
	\end{lstlisting}

	\textit{Membuat network service untuk fitur pencatatan kualitas air harian}
	\begin{lstlisting}
import 'dart:convert';
import 'package:fish/models/daily_water_model.dart';
import 'package:fish/service/url_api.dart';
import 'package:http/http.dart' as http;

class DailyWaterService {
    Future<List<DailyWater>> getPonds() async {
    var url = Uri.parse(Urls.dailyWater);
    var headers = {'Content-Type': 'application/json'};

    var response = await http.get(url, headers: headers);

    print(response.body);

    if (response.statusCode == 200) {
        var data = jsonDecode(response.body);
        List<DailyWater> ponds = [];

        for (var item in data) {
        ponds.add(DailyWater.fromJson(item));
        }

        print(ponds);

        return ponds;
    } else {
        throw Exception('Gagal Get Ponds!');
    }
    }
    
    Future<bool> postDailyWater({
    required String? pondId,
    required String? activationId,
    required String? ph,
    required String? numDo,
    String? week,
    required String? temperature,
    }) async {
    print({
        "pond_id": pondId.toString(),
        "pond_activation_id": activationId.toString(),
        "ph": ph,
        "do": numDo,
        "week": week,
        "temperature": temperature,
    });
    final response = await http.post(
        Uri.parse(Urls.dailyWater),
        headers: {
        "Content-Type": "application/x-www-form-urlencoded",
        },
        encoding: Encoding.getByName('utf-8'),
        body: {
        "pond_id": pondId.toString(),
        "pond_activation_id": activationId.toString(),
        "ph": ph,
        "week": week,
        "do": numDo,
        "temperature": temperature,
        },
    );

    if (response.statusCode == 200) {
        print(response.body);
        return true;
    } else {
        print(response.body);
        return false;
    }
    }

    Future<bool> editDailyWater(
        {required String? dailywaterId,
        required String? ph,
        required String? numDo,
        required String? temperature}) async {
    print({
        "ph": ph,
        "do": numDo,
        "temperature": temperature,
    });
    final response = await http.put(
        Uri.parse(Urls.dailyWaterbyid(dailywaterId)),
        headers: {
        "Content-Type": "application/x-www-form-urlencoded",
        },
        encoding: Encoding.getByName('utf-8'),
        body: {
        "ph": ph,
        "do": numDo,
        "temperature": temperature,
        },
    );

    if (response.statusCode == 200) {
        print(response.body);
        return true;
    } else {
        print(response.body);
        return false;
    }
    }

    Future<DailyWater> DeleteDatas({required String dailywaterId}) async {
    var url = Uri.parse(Urls.dailyWaterbyid(dailywaterId));
    var headers = {'Content-Type': 'application/json'};

    var response = await http.delete(url, headers: headers);

    print(response.body);

    if (response.statusCode == 200) {
        return DailyWater.fromJson(jsonDecode(response.body));
    } else {
        throw Exception('Gagal Get Ponds!');
    }
    }

    Future<List<DailyWater>> getDatas({required String dailywaterId}) async {
    var url = Uri.parse(Urls.dailyWaterbyid(dailywaterId));
    var headers = {'Content-Type': 'application/json'};

    var response = await http.get(url, headers: headers);

    print(response.body);

    if (response.statusCode == 200) {
        var data = jsonDecode(response.body);
        List<DailyWater> ponds = [];
        ponds.add(DailyWater.fromJson(data));

        print(ponds);

        return ponds;
    } else {
        throw Exception('Gagal Get Ponds!');
    }
    }
}        
	\end{lstlisting}

	\clearpage
	\section{Lampiran 10 Code Sprint 9 report}
	\textit{Code halaman list pencatatan kualitas air mingguan}
	\begin{lstlisting}
import 'package:fish/pages/component/daily_water_card.dart';
import 'package:fish/controllers/weeklywater/weekly_water_controller.dart';
import 'package:fish/pages/component/weekly_water_card.dart';
import 'package:fish/pages/weeklywater/weeklywater_avg.dart';
import 'package:fish/pages/weeklywater/weeklywater_entry_page.dart';
import 'package:flutter/material.dart';
import 'package:fish/theme.dart';
import 'package:get/get.dart';

class WeeklyWaterPage extends StatefulWidget {
    WeeklyWaterPage({Key? key}) : super(key: key);

    @override
    State<WeeklyWaterPage> createState() => _WeeklyWaterPageState();
}

class _WeeklyWaterPageState extends State<WeeklyWaterPage> {
    final WeeklyWaterController controller = Get.put(WeeklyWaterController());

    @override
    void initState() {
    super.initState();
    controller.getWeeklyWaterData(context);
    }

    @override
    void dispose() {
    controller.postDataLog(controller.fitur);
    super.dispose();
    }

    @override
    Widget build(BuildContext context) {
    Widget fishDataRecap() {...}

    Widget listWeeklyWater() {...}

    Widget emptyList() {...}

    return Obx(() {
        if (controller.isLoading.value == false) {
        return Scaffold(
            floatingActionButton: FloatingActionButton(
            onPressed: () {
                Get.to(() => WeeklyWaterEntryPage(), arguments: {
                "pond": controller.pond,
                "activation": controller.activation
                });
                controller.postDataLog(controller.fitur);
            },
            backgroundColor: primaryColor,
            child: const Icon(Icons.add),
            ),
            backgroundColor: backgroundColor1,
            body: ListView(
            children: [
                fishDataRecap(),
                controller.listWeeklyWater.isEmpty
                    ? emptyList()
                    : listWeeklyWater(),
                SizedBox(
                height: 10,
                )
            ],
            ),
        );
        } else {
        return Center(
            child: CircularProgressIndicator(
            color: secondaryColor,
            ),
        );
        }
    });
    }
}        
	\end{lstlisting}

	\textit{Code untuk halaman entry kualitas air mingguan}
	\begin{lstlisting}
        import 'package:fish/controllers/weeklywater/weekly_water_entry_controller.dart';
        import 'package:fish/controllers/weeklywater/weekly_water_controller.dart';
        import 'package:flutter/material.dart';
        import 'package:fish/theme.dart';
        import 'package:flutter/services.dart';
        import 'package:get/get.dart';
        
        class WeeklyWaterEntryPage extends StatelessWidget {
          WeeklyWaterEntryPage({Key? key}) : super(key: key);
        
          final WeeklyWaterEntryController controller =
              Get.put(WeeklyWaterEntryController());
        
          final WeeklyWaterController weeklyWaterControlller =
              Get.put(WeeklyWaterController());
        
          @override
          Widget build(BuildContext context) {
            Widget amoniaInput() {...}
        
            Widget flocInput() {...}
        
            Widget nitriteInput() {...}
        
            Widget nitrateInput() {...}
        
            Widget hardnessInput() {...}
        
            Widget submitButton() {...}
        
            return Obx(() {
              if (controller.isLoading.value == false) {
                return Scaffold(
                  appBar: AppBar(
                    backgroundColor: backgroundColor2,
                    title: const Text("Entry Kondisi Air Harian"),
                  ),
                  backgroundColor: backgroundColor1,
                  body: ListView(
                    children: [
                      flocInput(),
                      amoniaInput(),
                      nitriteInput(),
                      nitrateInput(),
                      hardnessInput(),
                      submitButton(),
                      SizedBox(
                        height: 8,
                      )
                    ],
                  ),
                );
              } else {
                return Center(
                  child: CircularProgressIndicator(
                    color: secondaryColor,
                  ),
                );
              }
            });
          }
        }
        
	\end{lstlisting}

    \textit{Code untuk halaman detail kualitas air mingguan}
	\begin{lstlisting}
        import 'package:fish/controllers/weeklywater/weekly_water_detail_controller.dart';
import 'package:flutter/material.dart';
import 'package:fish/theme.dart';
import 'package:get/get.dart';

class WeeklyWaterDetailPage extends StatelessWidget {
  const WeeklyWaterDetailPage({Key? key}) : super(key: key);

  @override
  Widget build(BuildContext context) {
    final WeeklyWaterDetailController controller =
        Get.put(WeeklyWaterDetailController());

    Widget airMingguanDataRecap() {...}

    Widget detail() {...}

    Widget titleRecap() {...}

    Widget dataAirMingguan() {...}

    Widget detailAirMingguan() {...}

    return Obx(() {
      if (controller.isLoading.value == false) {
        return Scaffold(
          appBar: AppBar(
            backgroundColor: backgroundColor2,
            title: const Text("Detail Kondisi Air Harian"),
          ),
          backgroundColor: backgroundColor1,
          body: ListView(
            children: [
              airMingguanDataRecap(),
              detail(),
              titleRecap(),
              dataAirMingguan(),
              detailAirMingguan(),
              SizedBox(
                height: 10,
              )
            ],
          ),
        );
      } else {
        return Center(
          child: CircularProgressIndicator(
            color: secondaryColor,
          ),
        );
      }
    });
  }
}   
	\end{lstlisting}

	\textit{Membuat model class fitur pencatatan kualitas air mingguan}
	\begin{lstlisting}
import 'package:flutter/material.dart';
import 'package:intl/intl.dart';

class WeeklyWater {
    String? id;
    String? pondId;
    String? activationId;
    num? ammonia;
    num? floc;
    num? nitrite;
    num? nitrate;
    num? hardness;
    num? week;
    String? floc_desc;
    String? ammonia_desc;
    String? hardness_desc;
    String? nitrate_desc;
    String? nitrite_desc;
    String? weeklywater_at;

    WeeklyWater(
        {required this.id,
        this.pondId,
        this.activationId,
        this.floc,
        this.floc_desc,
        this.ammonia,
        this.ammonia_desc,
        this.nitrate,
        this.nitrate_desc,
        this.nitrite,
        this.nitrite_desc,
        this.hardness,
        this.hardness_desc,
        this.week,
        this.weeklywater_at});

    factory WeeklyWater.fromJson(Map<String, dynamic> json) {
    return WeeklyWater(
        id: json['_id'],
        pondId: json['pond_id'],
        activationId: json['pond_activation_id'],
        floc: json['floc'],
        nitrate: json['nitrate'],
        nitrite: json['nitrite'],
        ammonia: json['ammonia'],
        hardness: json['hardness'],
        floc_desc: json['floc_desc'],
        nitrate_desc: json['nitrate_desc'],
        nitrite_desc: json['nitrite_desc'],
        ammonia_desc: json['ammonia_desc'],
        hardness_desc: json['hardness_desc'],
        week: json['week'],
        weeklywater_at: json['weeklywater_at']);
    }
    String getGmtToNormalDate() {
    String stringDate = weeklywater_at!;
    DateTime dateTime = DateFormat("yyyy-MM-dd hh:mm:ss").parse(stringDate);
    String newStringDate = DateFormat("dd-MM-yyyy").format(dateTime);
    return newStringDate;
    }

    String getDayNameDate() {
    String stringDate = weeklywater_at!;
    DateTime dateTime = DateFormat("yyyy-MM-dd hh:mm:ss").parse(stringDate);
    String newStringDate = DateFormat("EEEE").format(dateTime);
    return newStringDate;
    }
}        
	\end{lstlisting}
	
	\textit{Membuat network service untuk fitur pencatatan kualitas air mingguan}
	\begin{lstlisting}
import 'dart:convert';
import 'package:fish/models/weeklywater_model.dart';
import 'package:fish/service/url_api.dart';
import 'package:http/http.dart' as http;

class WeeklyWaterService {
    Future<List<WeeklyWater>> getDatas() async {
    var url = Uri.parse(Urls.weeklyWater);
    var headers = {'Content-Type': 'application/json'};

    var response = await http.get(url, headers: headers);

    print(response.body);

    if (response.statusCode == 200) {
        var data = jsonDecode(response.body);
        List<WeeklyWater> ponds = [];

        for (var item in data) {
        ponds.add(WeeklyWater.fromJson(item));
        }

        print(ponds);

        return ponds;
    } else {
        throw Exception('Gagal Get Ponds!');
    }
    }

    Future<bool> postWeeklyWater({
    required String? pondId,
    required String? activationId,
    required String? floc,
    String? ammonia,
    String? nitrite,
    String? nitrate,
    String? hardness,
    String? week,
    }) async {
    print({
        "pond_id": pondId.toString(),
        "pond_activation_id": activationId.toString(),
        "floc": floc,
        "nitrite": nitrate,
        "nitrate": nitrate,
        "ammonia": ammonia,
        "hardness": hardness,
        "week": week,
    });
    final response = await http.post(
        Uri.parse(Urls.weeklyWater),
        headers: {
        "Content-Type": "application/x-www-form-urlencoded",
        },
        encoding: Encoding.getByName('utf-8'),
        body: {
        "pond_id": pondId.toString(),
        "pond_activation_id": activationId.toString(),
        "floc": floc,
        "nitrite": nitrate,
        "nitrate": nitrate,
        "ammonia": ammonia,
        "hardness": hardness,
        "week": week,
        },
    );

    if (response.statusCode == 200) {
        print(response.body);
        return true;
    } else {
        print(response.body);
        return false;
    }
    }
}        
	\end{lstlisting}

	\clearpage
	\section{Lampiran 11 Code Sprint 10 report}
	\textit{Code halaman rekapitulasi sortir}
	\begin{lstlisting}
import 'package:fish/pages/component/transfer_card.dart';
import 'package:fish/controllers/fish_transfer/fish_transfer_list_controller.dart';
import 'package:fish/pages/fish_transfer/fish_transfer_entry_page.dart';
import 'package:flutter/material.dart';
import 'package:fish/theme.dart';
import 'package:get/get.dart';

import 'new_fish_transfer_entry_page.dart';

class FishTransferListPage extends StatefulWidget {
    FishTransferListPage({Key? key}) : super(key: key);

    @override
    State<FishTransferListPage> createState() => _FishTransferListPageState();
}

class _FishTransferListPageState extends State<FishTransferListPage> {
    final TransferController controller = Get.put(TransferController());

    @override
    void initState() {
    super.initState();

    controller.getTransfertData(context);
    }

    @override
    void dispose() {
    controller.postDataLog(controller.fitur);
    super.dispose();
    }

    @override
    Widget build(BuildContext context) {...}

    Widget listTransfer() {...}

    Widget emptyListTransfer() {...}

    return Obx(() {
        if (controller.isLoading.value == false) {
        return Scaffold(
            floatingActionButton: FloatingActionButton(
            onPressed: () {
                Get.to(() => const NewFishTransferEntryPage(), arguments: {
                "pond": controller.pond,
                "activation": controller.activation
                });
                controller.postDataLog(controller.fitur);
            },
            backgroundColor: primaryColor,
            child: const Icon(Icons.add),
            ),
            backgroundColor: backgroundColor1,
            body: ListView(
            children: [
                fishDataRecap(),
                controller.listTransfer.isEmpty
                    ? emptyListTransfer()
                    : listTransfer(),
                SizedBox(
                height: 10,
                )
            ],
            ),
        );
        } else {
        return Center(
            child: CircularProgressIndicator(
            color: secondaryColor,
            ),
        );
        }
    });
    }
     
	\end{lstlisting}
	
	\textit{Code untuk halaman entry sortir}
	\begin{lstlisting}
        import 'package:fish/models/fish_model.dart';
        import 'package:fish/pages/component/treatment_berat_input_card.dart';
        import 'package:fish/pages/treatment/treatment_entry_controller.dart';
        import 'package:fish/controllers/fish_transfer/fish_transfer_entry_controller.dart';
        import 'package:fish/controllers/fish_transfer/pond_list_item_controller.dart';
        import 'package:flutter/material.dart';
        import 'package:fish/pages/pond/detail_pond_controller.dart';
        import 'package:fish/theme.dart';
        
        import 'package:fish/pages/component/deactivation_list_input.dart';
        import 'package:flutter/services.dart';
        import 'package:get/get.dart';
        
        import '../../controllers/fish_transfer/fish_transfer_list_controller.dart';
        import '../component/deactivation_with_fish_transfer_input.dart';
        import '../component/fish_transfer_input.dart';
        
        class FishTransferEntryPage extends StatefulWidget {
          FishTransferEntryPage({Key? key}) : super(key: key);
          @override
          State<FishTransferEntryPage> createState() => _FishTransferEntryPageState();
        }
        
        class _FishTransferEntryPageState extends State<FishTransferEntryPage> {
          final FishTransferEntryController controller =
              Get.put(FishTransferEntryController());
        
          final TransferController fishTransferController =
              Get.put(TransferController());
        
          final PondListController getpondlistcontroller =
              Get.put(PondListController());
        
          final pageController = PageController(initialPage: 0);
          @override
          void dispose() {
            controller.descController.clear();
            controller.sampleLongController.clear();
            controller.sampleWeightController.clear();
            controller.leleAmountActivationController.clear();
            controller.leleAmountController.clear();
            controller.leleAmountDeactivationController.clear();
            controller.leleWeightActivationController.clear();
            controller.leleWeightController.clear();
            controller.leleWeightDeactivationController.clear();
            controller.masAmountActivationController.clear();
            controller.masAmountController.clear();
            controller.masAmountDeactivationController.clear();
            controller.masWeightActivationController.clear();
            controller.masWeightController.clear();
            controller.masWeightDeactivationController.clear();
            controller.patinAmountActivationController.clear();
            controller.patinAmountController.clear();
            controller.patinAmountDeactivationController.clear();
            controller.patinWeightActivationController.clear();
            controller.patinWeightController.clear();
            controller.patinWeightDeactivationController.clear();
            controller.nilaMerahAmountActivationController.clear();
            controller.nilaMerahAmountController.clear();
            controller.nilaMerahAmountDeactivationController.clear();
            controller.nilaMerahWeightActivationController.clear();
            controller.nilaMerahWeightController.clear();
            controller.nilaMerahWeightDeactivationController.clear();
            controller.nilaHitamAmountActivationController.clear();
            controller.nilaHitamAmountController.clear();
            controller.nilaHitamAmountDeactivationController.clear();
            controller.nilaHitamWeightActivationController.clear();
            controller.nilaHitamWeightController.clear();
            controller.nilaHitamWeightDeactivationController.clear();
            pageController.dispose();
            controller.postDataLog(controller.fitur);
            super.dispose();
          }
        
          void initState() {
            super.initState();
            // WidgetsBinding.instance.addPostFrameCallback((timeStamp) async {
            //   await controller.getPondActivations(
            //       pondId: controller.pond.id.toString());
            // });
            // controller.getHarvestedBool(controller.activation);
            controller.getPondsData(controller.methodController.toString());
            controller.getHarvestedBool(controller.activation);
          }
        
          @override
          Widget build(BuildContext context) {
            Widget sampleWeightInput() {...}
        
            Widget sampleLongInput() {...}
        
            Widget destinationPondInput() {...}
        
            Widget transferMethodInput() {...}
        
            Widget checkBoxFishTransfer() {...}
        
            Widget leleInput() {...}
        
            Widget nilaMerahInput() {...}
        
            Widget nilaHitamInput() {...}
        
            Widget patinInput() {...}
        
            Widget masInput() {...}
        
        //input aktivasi
            Widget checkBoxFish() {...}
        
            Widget waterHeightInput() {...}
        
            Widget leleInputActivation() {...}
        
            Widget nilaMerahInputActivation() {...}
        
            Widget nilaHitamInputActivation() {...}
        
            Widget patinInputActivation() {...}
        
            Widget masInputActivation() {...}
        
            Widget submitButton() {...}
        
            Widget submitKeringButton() {...}
        
            Widget previousSubmitButton() {...}
        
            Widget previousNextButton() {...}
        
            Widget nextButton() {...}
        
            Widget previousButton() {...}
        
            Widget deactivationInput() {...}
        
            Widget destinationnNotActiveTransfer() {...}
        
            Widget deactivationTransfer() {...}
        
            return Scaffold(
                appBar: AppBar(
                  backgroundColor: backgroundColor2,
                  title: const Text("Entry Sortir"),
                ),
                backgroundColor: backgroundColor1,
                body: PageView(
                  physics: const NeverScrollableScrollPhysics(),
                  controller: pageController,
                  children: [
                    Obx(() {
                      return ListView(
                        children: [
                          transferMethodInput(),
                          destinationPondInput(),
                          checkBoxFishTransfer(),
                          controller.isNilaHitamInput == true
                              ? nilaHitamInput()
                              : Container(),
                          controller.isNilaMerahInput == true
                              ? nilaMerahInput()
                              : Container(),
                          controller.isLeleInput == true ? leleInput() : Container(),
                          controller.isPatinInput == true ? patinInput() : Container(),
                          controller.isMasInput == true ? masInput() : Container(),
                          sampleLongInput(),
                          sampleWeightInput(),
                          controller.methodController.selected.value == "basah"
                              ? submitButton()
                              : nextButton(),
                          SizedBox(
                            height: 8,
                          )
                        ],
                      );
                    }),
                    Obx(() {
                      return ListView(
                        children: [
                          controller.destinationIsActive == false
                              ? destinationnNotActiveTransfer()
                              : deactivationInput(),
                          controller.destinationIsActive == false
                              ? checkBoxFish()
                              : deactivationInput(),
                          controller.isNilaHitamActivation == true
                              ? nilaHitamInputActivation()
                              : Container(),
                          controller.isNilaMerahActivation == true
                              ? nilaMerahInputActivation()
                              : Container(),
                          controller.isLeleActivation == true
                              ? leleInputActivation()
                              : Container(),
                          controller.isPatinActivation == true
                              ? patinInputActivation()
                              : Container(),
                          controller.isMasActivation == true
                              ? masInputActivation()
                              : Container(),
                          controller.destinationIsActive == false
                              ? waterHeightInput()
                              : Container(),
                          controller.destinationIsActive == false
                              ? previousNextButton()
                              : previousSubmitButton(),
                          SizedBox(
                            height: 8,
                          )
                        ],
                      );
                    }),
                    ListView(
                      children: [
                        deactivationTransfer(),
                        deactivationInput(),
                        previousSubmitButton(),
                        SizedBox(
                          height: 8,
                        )
                      ],
                    ),
                  ],
                ));
          }
        }
        

	\end{lstlisting}

	\textit{Code untuk halaman detail sortir}
	\begin{lstlisting}
        import 'package:fish/pages/treatment/treatment_detail_controller.dart';
        import 'package:flutter/material.dart';
        import 'package:fish/theme.dart';
        import 'package:get/get.dart';
        import '../../controllers/fish_transfer/fish_transfer_detail_controller.dart';
        
        class DetailSortirPage extends StatelessWidget {
          const DetailSortirPage({Key? key}) : super(key: key);
        
          @override
          Widget build(BuildContext context) {
            final SortirDetailController controller = Get.put(SortirDetailController());
        
            Widget transferDataRecap() {...}
        
            Widget detail() {...}
        
            Widget titleRecap() {...}
        
            Widget dataTreatment() {...}
        
            Widget detailSortir() {...}
        
            return Obx(() {
              if (controller.isLoading.value == false) {
                return Scaffold(
                  appBar: AppBar(
                    backgroundColor: backgroundColor2,
                    title: const Text("Detail Sortir"),
                  ),
                  backgroundColor: backgroundColor1,
                  body: ListView(
                    children: [
                      transferDataRecap(),
                      detail(),
                      titleRecap(),
                      dataTreatment(),
                      detailSortir(),
                      SizedBox(
                        height: 10,
                      )
                    ],
                  ),
                );
              } else {
                return Center(
                  child: CircularProgressIndicator(
                    color: secondaryColor,
                  ),
                );
              }
            });
          }
        }        
	\end{lstlisting}

	\textit{Membuat model class fitur sortir ikan}
	\begin{lstlisting}
        import 'package:fish/models/fish_harvested.dart';
        import 'package:intl/intl.dart';
        import 'package:fish/models/fish_model.dart';
        
        class FishTransfer {
          String? id;
          String? origin_pond_id;
          String? origin_activation_id;
          String? destination_pond_id;
          String? destination_activation_id;
          String? transfer_method;
          String? transfer_type;
          num? sampleLong;
          num? sampleWeight;
          List<FishHarvest>? fishTransfer;
          String? transferAt;
        
          FishTransfer({
            required this.id,
            required this.origin_pond_id,
            required this.origin_activation_id,
            required this.destination_pond_id,
            required this.destination_activation_id,
            required this.transfer_type,
            required this.transfer_method,
            required this.sampleLong,
            required this.sampleWeight,
            required this.fishTransfer,
            this.transferAt,
          });
        
          factory FishTransfer.fromJson(Map<String, dynamic> json) {
            print(json);
            return FishTransfer(
              id: json['_id'],
              origin_pond_id: json['origin_pond_id'],
              origin_activation_id: json['origin_activation_id'],
              destination_pond_id: json['destination_pond_id'],
              destination_activation_id: json['destination_activation_id'],
              transfer_method: json['transfer_method'],
              transfer_type: json['transfer_type'],
              sampleLong: json['sample_long'],
              sampleWeight: json['sample_weight'],
              fishTransfer: FishHarvest.fromJsonList(json['fish']),
              transferAt: json['transfer_at'],
            );
          }
        
          String getGmtToNormalDate() {
            String stringDate = transferAt!;
            DateTime dateTime = DateFormat("yyyy-MM-dd hh:mm:ss").parse(stringDate);
            String newStringDate = DateFormat("dd-MM-yyyy").format(dateTime);
            return newStringDate;
          }
        }          
	\end{lstlisting}

	\textit{Membuat network service untuk fitur sortir ikan}
	\begin{lstlisting}
        import 'dart:convert';
        import 'dart:developer';
        import 'package:fish/models/fish_transfer_model.dart';
        import 'package:fish/service/url_api.dart';
        import 'package:flutter/material.dart';
        import 'package:http/http.dart' as http;
        
        class FishTransferService {
          Future<List<FishTransfer>> getFishTransferList() async {
            var url = Uri.parse(Urls.fishtransfer);
            var headers = {'Content-Type': 'application/json'};
        
            var response = await http.get(url, headers: headers);
        
            print(response.body);
        
            if (response.statusCode == 200) {
              var data = jsonDecode(response.body);
              List<FishTransfer> transferhistory = [];
        
              for (var item in data) {
                transferhistory.add(FishTransfer.fromJson(item));
              }
              // Treatment treatment = Treatment.fromJson(data[0]);
              // print(data[1]);
        
              print("ini leght transfer ${transferhistory.length}");
              return transferhistory;
            } else {
              throw Exception('Gagal Get Products!');
            }
          }
        
          Future<bool> postFishTransfer({
            required String? origin_pond_id,
            required String? destination_pond_id,
            required String? transfer_method,
            required String? transfer_type,
            required String? sample_weight,
            required String? sample_long,
            required List? fish,
          }) async {
            print({
              "origin_pond_id": origin_pond_id.toString(),
              "destination_pond_id": destination_pond_id.toString(),
              "transfer_method": transfer_method,
              "transfer_type": transfer_type,
              "sample_long": sample_long,
              "sample_weight": sample_weight,
              "fish": fish.toString()
            });
            final response = await http.post(
              Uri.parse(Urls.fishtransfer),
              headers: {
                "Content-Type": "application/x-www-form-urlencoded",
              },
              encoding: Encoding.getByName('utf-8'),
              body: {
                "origin_pond_id": origin_pond_id.toString(),
                "destination_pond_id": destination_pond_id.toString(),
                "transfer_method": transfer_method,
                "transfer_type": transfer_type,
                "sample_long": sample_long,
                "sample_weight": sample_weight,
                "fish": fish.toString()
              },
            );
        
            if (response.statusCode == 200) {
              print(response.body);
              return true;
            } else {
              print(response.body);
              return false;
            }
          }
        
          Future<bool> postFishTransferKering(
              {required String? origin_pond_id,
              required String? destination_pond_id,
              required String? transfer_method,
              required String? transfer_type,
              required String? sample_weight,
              required String? sample_long,
              required List? fish,
              required List? fishstock,
              required List? fishharvested,
              required num? total_fish_harvested,
              required num? total_weight_harvested,
              String? water_level}) async {
            print({
              "origin_pond_id": origin_pond_id.toString(),
              "destination_pond_id": destination_pond_id.toString(),
              "transfer_method": transfer_method,
              "transfer_type": transfer_type,
              "sample_long": sample_long,
              "sample_weight": sample_weight,
              "fish": fish.toString(),
              "fish_stock": fishstock.toString(),
              "fish_harvested": fishharvested.toString(),
              "total_weight_harvested": total_weight_harvested.toString(),
              "total_fish_harvested": total_fish_harvested.toString(),
              "water_level": water_level
            });
            final response = await http.post(
              Uri.parse(Urls.fishtransfer),
              headers: {
                "Content-Type": "application/x-www-form-urlencoded",
              },
              encoding: Encoding.getByName('utf-8'),
              body: {
                "origin_pond_id": origin_pond_id.toString(),
                "destination_pond_id": destination_pond_id.toString(),
                "transfer_method": transfer_method,
                "transfer_type": transfer_type,
                "sample_long": sample_long,
                "sample_weight": sample_weight,
                "fish": fish.toString(),
                "fish_stock": fishstock.toString(),
                "fish_harvested": fishharvested.toString(),
                "total_weight_harvested": total_weight_harvested.toString(),
                "total_fish_harvested": total_fish_harvested.toString(),
                "water_level": water_level
              },
            );
        
            if (response.statusCode == 200) {
              print(response.body);
              return true;
            } else {
              print(response.body);
              return false;
            }
          }
        
          Future<bool> postTransfer(
              {required String origin_pond_id,
              required String transfer_method,
              required String total_fish_harvested,
              required String total_weight_harvested,
              required String amountUndersize,
              required String amountOversize,
              required String amountNormal,
              required String sampleWeight,
              required String sampleLong,
              required String sampleAmount,
              required List<dynamic> transferList,
              required List<dynamic> fishDeath,
              required BuildContext ctx}) async {
            List<dynamic> transferListPost = [];
            for (var i in transferList) {
              final fish = [];
              for (var j in i["fish"]) {
                var k = json.decode(j);
                print("ini daata for ${k["type"]}");
                final datafish = {
                  "type": k["type"],
                  "amount": int.parse(k["amount"]),
                  "weight": double.parse(k["weight"])
                };
                fish.add(datafish);
                print("ini fish baru $fish");
              }
              final datas = {
                "destination_pond_id": i["destination_pond_id"],
                "status": i["status"],
                "fish": fish,
                "sample_weight": double.parse(i["sample_weight"]),
                "sample_long": double.parse(i['sample_long']),
                "transfer_type": i["transfer_type"],
                if (i["status"] == "isNotActivated") ...{
                  "water_level": int.parse(i["water_level"])
                }
              };
              transferListPost.add(datas);
            }
            print({
              "origin_pond_id": origin_pond_id.toString(),
              "fish_sort_type": transfer_method,
              "total_fish_harvested": total_fish_harvested,
              "total_weight_harvested": total_weight_harvested,
              "sample_long": sampleLong,
              "sample_amount": sampleAmount,
              "sample_weight": sampleWeight,
              "amount_oversize": amountOversize,
              "amount_undersized": amountUndersize,
              "amount_normal": amountNormal,
              "transfer_list": json.encode(transferListPost),
              "fish_death": json.encode(fishDeath)
            });
        
            final response = await http.post(
              Uri.parse(Urls.newfishtransfer),
              headers: {
                "Content-Type": "application/x-www-form-urlencoded",
              },
              encoding: Encoding.getByName('utf-8'),
              body: {
                "origin_pond_id": origin_pond_id.toString(),
                "fish_sort_type": transfer_method,
                "total_fish_harvested": total_fish_harvested,
                "total_weight_harvested": total_weight_harvested,
                "sample_long": sampleLong,
                "sample_amount": sampleAmount,
                "sample_weight": sampleWeight,
                "amount_oversize": amountOversize,
                "amount_undersized": amountUndersize,
                "amount_normal": amountNormal,
                "transfer_list": json.encode(transferListPost),
                if (transfer_method == "kering") ...{
                  "fish_death": json.encode(fishDeath)
                }
              },
            );
        
            if (response.statusCode == 200) {
              print(response.body);
              final snackBar = SnackBar(
                content: const Text('Sortir Ikan Berhasil!'),
                backgroundColor: Colors.green,
              );
              ScaffoldMessenger.of(ctx).showSnackBar(snackBar);
              return true;
            } else {
              print(response.body);
              final snackBar = SnackBar(
                content: Text(response.body.toString()),
                backgroundColor: Colors.red,
              );
              ScaffoldMessenger.of(ctx).showSnackBar(snackBar);
        
              print("gagal post");
        
              return false;
            }
          }
        }        
	\end{lstlisting}

	\clearpage
	\section{Lampiran 12 Code Sprint 11 report}

	
	\textit{Code halaman login}
	\begin{lstlisting}
        import 'dart:convert';
        import 'dart:developer';
        
        import 'package:fish/controllers/authentication/login_controller.dart';
        import 'package:fish/pages/authentication/register_page.dart';
        // import 'package:rflutter_alert/rflutter_alert.dart';
        import 'package:fish/pages/dashboard.dart';
        import 'package:flutter/material.dart';
        import 'package:fish/theme.dart';
        import 'package:flutter/services.dart';
        import 'package:get/get.dart';
        import 'package:shared_preferences/shared_preferences.dart';
        import 'package:fish/service/url_api.dart';
        import 'package:shared_preferences/shared_preferences.dart';
        
        import 'package:http/http.dart' as http;
        
        import '../component/login_card_input.dart';
        
        class LoginPage extends StatefulWidget {
          LoginPage({Key? key}) : super(key: key);
          @override
          State<LoginPage> createState() => _LoginPageState();
        }
        
        class _LoginPageState extends State<LoginPage> {
          late SharedPreferences prefs;
          final LoginController controller = Get.put(LoginController());
          @override
          void initState() {
            super.initState();
            initSharedPrefs();
          }
        
          void initSharedPrefs() async {
            prefs = await SharedPreferences.getInstance();
            inspect(prefs);
          }
        
          void login() async {
            final response = await http.post(
              Uri.parse(Urls.authentication),
              headers: {
                "Content-Type": "application/x-www-form-urlencoded",
              },
              encoding: Encoding.getByName('utf-8'),
              body: {
                "username": controller.usernameController.text,
                "password": controller.passwordController.text,
              },
            );
            var data = jsonDecode(response.body);
            print(response.body);
            if (response.statusCode == 200) {
              var myToken = data['access_token'];
              var identity = data['identity'];
              prefs.setString(
                'token',
                myToken,
              );
              prefs.setString('identity', identity.toString());
              // prefs.setString('identity', identity);
              Navigator.pushReplacement(
                  context, MaterialPageRoute(builder: (context) => DashboardPage()));
              controller.usernameController.clear();
              controller.passwordController.clear();
            } else {
              showDialog<String>(
                  context: context,
                  builder: (BuildContext context) => AlertDialog(
                        title: const Text('Login Error',
                            style: TextStyle(color: Colors.red)),
                        content: const Text(
                          'BreederID/Password salah',
                          style: TextStyle(color: Colors.white),
                        ),
                        backgroundColor: backgroundColor1,
                        shape: RoundedRectangleBorder(
                            borderRadius: BorderRadius.all(Radius.circular(16.0))),
                        actions: <Widget>[
                          TextButton(
                            onPressed: () => Navigator.pop(context, 'OK'),
                            child: const Text('OK'),
                          ),
                        ],
                      ));
            }
          }
        
          @override
          Widget build(BuildContext context) {
        
            Widget formInput() {...}
        
            Widget logo() {...}
        
            Widget footer() {...}
        
            Widget submitButton() {...}
        
            return Obx(() {
              if (controller.isLoading.value == false) {
                return Scaffold(
                  backgroundColor: backgroundColor2,
                  body: ListView(
                    children: [
                      SizedBox(
                        height: 20,
                      ),
                      logo(),
                      formInput(),
                      SizedBox(
                        height: 16,
                      ),
                      footer(),
                      // submitButton(),
                      SizedBox(
                        height: 8,
                      )
                    ],
                  ),
                );
              } else {
                return Center(
                  child: CircularProgressIndicator(
                    color: secondaryColor,
                  ),
                );
              }
            });
          }
        }          
	\end{lstlisting}
	
	\textit{Code untuk halaman register}
	\begin{lstlisting}
        import 'dart:convert';
        import 'dart:developer';
        
        import 'package:fish/controllers/authentication/register_controller.dart';
        import 'package:fish/pages/dashboard.dart';
        import 'package:flutter/material.dart';
        import 'package:fish/theme.dart';
        import 'package:flutter/services.dart';
        import 'package:get/get.dart';
        import 'package:shared_preferences/shared_preferences.dart';
        import 'package:fish/service/url_api.dart';
        import 'package:shared_preferences/shared_preferences.dart';
        
        import 'package:http/http.dart' as http;
        
        import '../component/login_card_input.dart';
        import '../component/register_input.dart';
        import '../component/register_input_next.dart';
        import 'login_page.dart';
        
        class RegisterPage extends StatefulWidget {
          RegisterPage({Key? key}) : super(key: key);
          @override
          State<RegisterPage> createState() => _RegisterPageState();
        }
        
        class _RegisterPageState extends State<RegisterPage> {
          late SharedPreferences prefs;
          final RegisterController controller = Get.put(RegisterController());
        
          final pageController = PageController(initialPage: 0);
          @override
          void dispose() {
            pageController.dispose();
        
            super.dispose();
          }
        
          void initState() {
            super.initState();
            initSharedPrefs();
            controller.getFarmData();
          }
        
          void initSharedPrefs() async {
            prefs = await SharedPreferences.getInstance();
            inspect(prefs);
          }
        
          void register() async {
            final response = await http.post(
              Uri.parse(Urls.register),
              headers: {
                "Content-Type": "application/x-www-form-urlencoded",
              },
              encoding: Encoding.getByName('utf-8'),
              body: {
                "username": controller.usernameController.text,
                "password": controller.passwordController.text,
                "nik": controller.nikController.text,
                "name": controller.nameController.text,
                "phone": controller.phoneController.text,
                "hasFarm": controller.hasFarmController.selected.value,
                "farm_id": controller.farmIdSelected,
                "farm_name": controller.farmnameController.text,
                "breeder": controller.breedercountController.text,
                "address": controller.addressController.text,
                "coordinate": controller.coordinateController.text
              },
            );
            var data = jsonDecode(response.body);
            print(response.body);
            if (response.statusCode == 200) {
              var myToken = data['access_token'];
              var identity = data['identity'];
              prefs.setString(
                'token',
                myToken,
              );
              prefs.setString('identity', identity.toString());
              Navigator.pushReplacement(
                  context, MaterialPageRoute(builder: (context) => DashboardPage()));
              print(response.body);
              controller.usernameController.clear();
              controller.passwordController.clear();
              controller.addressController.clear();
              controller.breedercountController.clear();
              controller.coordinateController.clear();
              controller.phoneController.clear();
              controller.nameController.clear();
              controller.farmnameController.clear();
              controller.nikController.clear();
            } else {
              showDialog<String>(
                  context: context,
                  builder: (BuildContext context) => AlertDialog(
                        title: const Text('Register Error',
                            style: TextStyle(color: Colors.red)),
                        content: const Text(
                          'BreederID Sudah Digunakan',
                          style: TextStyle(color: Colors.white),
                        ),
                        backgroundColor: backgroundColor1,
                        shape: RoundedRectangleBorder(
                            borderRadius: BorderRadius.all(Radius.circular(16.0))),
                        actions: <Widget>[
                          TextButton(
                            onPressed: () => Navigator.pop(context, 'OK'),
                            child: const Text('OK'),
                          ),
                        ],
                      ));
              print(response.body);
            }
          }
        
          @override
          Widget build(BuildContext context) {
        
            Widget formInput() {...}
        
            Widget form2Input() {...}
        
            Widget logo() {...}
        
            Widget footer() {...}
        
            Widget submitButton() {...}
        
            return Obx(() {
              if (controller.isLoading.value == false) {
                return Scaffold(
                    backgroundColor: backgroundColor2,
                    body: PageView(
                      physics: const NeverScrollableScrollPhysics(),
                      controller: pageController,
                      children: [
                        ListView(
                          children: [
                            SizedBox(
                              height: 10,
                            ),
                            footer(),
                            formInput(),
                            SizedBox(
                              height: 10,
                            ),
                          ],
                        ),
                        ListView(
                          children: [
                            SizedBox(
                              height: 10,
                            ),
                            footer(),
                            form2Input(),
                            SizedBox(
                              height: 10,
                            ),
                          ],
                        ),
                      ],
                    ));
              } else {
                return Center(
                  child: CircularProgressIndicator(
                    color: secondaryColor,
                  ),
                );
              }
            });
          }
        }        
	\end{lstlisting}

	\textit{Code untuk halaman profile}
	\begin{lstlisting}
        import 'dart:convert';
        import 'dart:developer';
        
        import 'package:fish/controllers/authentication/register_controller.dart';
        import 'package:fish/pages/dashboard.dart';
        import 'package:flutter/material.dart';
        import 'package:fish/theme.dart';
        import 'package:get/get.dart';
        import 'package:shared_preferences/shared_preferences.dart';
        import '../../controllers/authentication/profile_controller.dart';
        import 'login_page.dart';
        
        class ProfilePage extends StatelessWidget {
          ProfilePage({Key? key}) : super(key: key);
          @override
          final ProfileController controller = Get.put(ProfileController());
        
          @override
          Widget build(BuildContext context) {
            Widget logo() {...}
        
            Widget footer() {...}
        
            Widget profileList() {...}
        
            Widget logoutButton() {...}
        
            Widget submitButton() {...}
        
            return Obx(() {
              if (controller.isLoading.value == false) {
                return Scaffold(
                  backgroundColor: backgroundColor2,
                  body: ListView(
                    children: [
                      SizedBox(
                        height: 10,
                      ),
                      logo(),
                      SizedBox(
                        height: 16,
                      ),
                      profileList(),
                      logoutButton(),
                      SizedBox(
                        height: 8,
                      ),
                      // footer(),
                      // submitButton(),
                    ],
                  ),
                );
              } else {
                return Center(
                  child: CircularProgressIndicator(
                    color: secondaryColor,
                  ),
                );
              }
            });
          }
        }        
	\end{lstlisting}

	\textit{Membuat model class user}
	\begin{lstlisting}
        import 'package:flutter/material.dart';
        import 'package:intl/intl.dart';
        
        class Breeder {
          String? id;
          String? farm_id;
          String? name;
          String? username;
          String? farm_name;
          String? nik;
          String? phone;
          String? address;
        
          Breeder({
            required this.id,
            required this.farm_id,
            required this.name,
            required this.username,
            required this.farm_name,
            required this.address,
            required this.nik,
            required this.phone,
          });
        
          factory Breeder.fromJson(Map<String, dynamic> json) {
            return Breeder(
              id: json['_id'],
              farm_id: json['farm_id'],
              farm_name: json['farm_name'],
              address: json['address'],
              name: json['name'],
              username: json['username'],
              nik: json['nik'],
              phone: json['phone'],
            );
          }
        
          static DateTime stringToDate(String dateString) {
            DateTime parseDate = DateFormat("dd-MM-yyyy").parse(dateString);
            return parseDate;
          }
        }        
	\end{lstlisting}

	\textit{Membuat model class farm}
	\begin{lstlisting}
        import 'package:flutter/material.dart';
        import 'package:intl/intl.dart';
        
        class Farm {
          String? id;
          String? farm_name;
          String? breeder;
          String? address;
          String? coordinate;
        
          Farm({
            required this.id,
            required this.farm_name,
            required this.breeder,
            required this.address,
            required this.coordinate,
          });
        
          factory Farm.fromJson(Map<String, dynamic> json) {
            return Farm(
                id: json['_id'],
                farm_name: json['farm_name'],
                breeder: json['breeder'],
                address: json['address'],
                coordinate: json['coordinate']);
          }
        
          static DateTime stringToDate(String dateString) {
            DateTime parseDate = DateFormat("dd-MM-yyyy").parse(dateString);
            return parseDate;
          }
        }           
	\end{lstlisting}


 



 